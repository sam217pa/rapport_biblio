% Created 2016-01-14 Thu 15:51
\documentclass[9pt, oneside, twocolumn]{scrartcl}
                % Created 2016-01-23 Sat 17:11
\documentclass[9pt, oneside, twocolumn]{scrartcl}
                % Created 2016-01-23 Sat 17:11
\documentclass[9pt, oneside, twocolumn]{scrartcl}
                % Created 2016-01-23 Sat 17:11
\documentclass[9pt, oneside, twocolumn]{scrartcl}
                \input{/Users/samuelbarreto/.templates/journal.tex}
               
\usepackage[utf8]{inputenc}
\usepackage[T1]{fontenc}
\usepackage{graphicx}
\usepackage{longtable}
\usepackage{float}
\usepackage{hyperref}
\usepackage{wrapfig}
\usepackage{rotating}
\usepackage[normalem]{ulem}
\usepackage{amsmath}
\usepackage{textcomp}
\usepackage{marvosym}
\usepackage{wasysym}
\usepackage{amssymb}
\usepackage[scaled=0.9]{zi4}
\usepackage[x11names, dvipsnames]{xcolor}
\usepackage[protrusion=true, expansion=alltext, tracking=true, kerning=true]{microtype}
\usepackage{siunitx}
\usepackage[french, frenchb]{babel}
\author{Samuel BARRETO}
\date{\today}
\title{}
\hypersetup{
 pdfauthor={Samuel BARRETO},
 pdftitle={},
 pdfkeywords={},
 pdfsubject={},
 pdfcreator={Emacs 24.5.1 (Org mode 8.3.3)}, 
 pdflang={Frenchb}}
\begin{document}

\setcounter{tocdepth}{2}
\tableofcontents

\clearpage

\part{2016-01 Janvier}
\label{sec:orgheadline86}
\section{2016-01-11 Lundi}
\label{sec:orgheadline6}
\subsection{(3) PCR cible des gènes synthétiques}
\label{sec:orgheadline4}
Ajouté le \textit{[2016-01-11 Mon 09:50]}

La PCR du \textit{[2016-01-07 Thu]}, cible des gènes synthétiques des clones
transformants hétérozygotes, n'a pas fonctionné, vraisemblablement dû à un
problème dans le mix PCR (pas d'amplification dans le t+). On refait donc la
même PCR sur moins de clones transformants, dans l'idée de se faire la main,
avant de faire des PCR plus conséquentes, avec de plus grands volumes. 

\subsubsection{Programme PCR :}
\label{sec:orgheadline1}
\begin{center}
\begin{tabular}{lrl}
\toprule
Temps & température & 30 cycles\\
\midrule
3' & 94 & \\
50'' & 94 & x\\
50'' & 58 & x\\
4' & 72 & x\\
\bottomrule
\end{tabular}
\end{center}

\subsubsection{Volumes :}
\label{sec:orgheadline2}
\begin{center}
\begin{tabular}{lrr}
\toprule
 & 1-tube & 7-tubes\\
\midrule
ADN & 0.5 & 3.5\\
Taq & 0.5 & 3.5\\
amorce 1392 & 2.5 & 17.5\\
amorce 1073 & 2.5 & 17.5\\
MgCl2 & 1.5 & 10.5\\
dNTPs & 5.0 & 35.\\
Tampon 10X Taq & 5.0 & 35.\\
\midrule
eau qsp 50µL & 32.5 & 227.5\\
\bottomrule
\end{tabular}
\end{center}

\subsubsection{Contrôle}
\label{sec:orgheadline3}
\begin{figure}[htb]
\centering
\includegraphics[width=0.4\linewidth]{/Volumes/HDD/stage/img/20160111_0.JPG}
\end{figure}

Ahah, la migration a duré trop longtemps à priori… Le bleu était sorti, petite
réunion avec Franck non-prévue…

On voit quand même des bandes dans les quatre puits, mais pas dans le témoin
positif, avec ADN génomique d'\emph{Acinetobacter}. Les bandes sont à la position
attendue dans le gel (taille : 750bp.) 

\begin{itemize}
\item voir à changer le témoin positif.
\end{itemize}

\subsection{(D) aliquoté Taq-ozyme}
\label{sec:orgheadline5}
Ajouté le \textit{[2016-01-11 Mon 11:47]}

Aliquoté 10µL de Taq-ozyme en eppendorf, placé dans le tiroir commun du -20°C.

\section{2016-01-12 Mardi}
\label{sec:orgheadline18}
\subsection{(3) PCR cible des gènes synthétiques}
\label{sec:orgheadline10}
Ajouté le \textit{[2016-01-12 Tue 08:19]}

La PCR d'hier ayant fonctionné, le but est de refaire la même PCR sur l'ensemble
des clones montrant des polymorphismes que j'ai isolé, avant de pouvoir les
envoyer à séquencer.

\begin{itemize}
\item Penser à changer de témoin positif (peut-être utiliser un plasmide porteur de
l'une des constructions par exemple.)
\end{itemize}

\subsubsection{Programme PCR :}
\label{sec:orgheadline7}
\begin{center}
\begin{tabular}{lrl}
\toprule
Temps & température & 30 cycles\\
\midrule
3' & 94 & \\
50'' & 94 & x\\
50'' & 58 & x\\
50'' & 72 & x\\
4' & 72 & \\
\bottomrule
\end{tabular}
\end{center}

\subsubsection{Volumes :}
\label{sec:orgheadline8}
\begin{center}
\begin{tabular}{lrr}
\toprule
 & 1-tube & 27-tubes\\
\midrule
ADN & 0.5 & 13.5\\
Taq & 0.5 & 13.5\\
amorce 1392 & 2.5 & 67.5\\
amorce 1073 & 2.5 & 67.5\\
MgCl2 & 1.5 & 40.5\\
dNTPs & 5.0 & 135.\\
Tampon 10X Taq & 5.0 & 135.\\
\midrule
eau qsp 50µL & 32.5 & 877.5\\
\bottomrule
\end{tabular}
\end{center}

\subsubsection{Contrôle :}
\label{sec:orgheadline9}

\begin{figure}[htb]
\centering
\includegraphics[width=3cm]{/Volumes/HDD/stage/img/20160112_00.JPG}
\caption{Ladder un peu chargé. Attention au volume. Colonies 3 et 4 de pW5 et 3 de pS53 n'ont pas amplifié. Témoin positif migration sur le gel de \textit{[2016-01-12 Tue 16:05]}.}
\end{figure}

\begin{itemize}
\item $\square$ refaire la PCR pour les deux clones pW5 et le clone de pS53 qui n'a pas
fonctionné.
\item $\square$ faire migrer le témoin positif.
\end{itemize}

\subsection{(2) PCR cible de l'ancre}
\label{sec:orgheadline15}
Ajouté le \textit{[2016-01-12 Tue 14:15]}

Chez les différents clones montrant des néomutations, on veut séquencer l'ancre,
pour vérifier que le taux de mutation dans la tract de conversion est bien
supérieur au taux de mutation dans une zone parfaitement homologue. L'hypothèse,
floue pour l'instant, est que le système de réparation des mésappariemments est
"saturé", et qu'il introduit des erreurs. (Ces mutations semblent biaisées vers
GC d'après les données dont on dispose pour l'instant.)

Le but est ici de réaliser une PCR simple sur quelques candidats, avant de
passer aux grands volumes, envoyés à séquencer. Les conditions de PCR sont
tirées du cahier de Florence, page 146, daté du \textit{[2015-06-25 Thu]}. 

\subsubsection{Candidats sélectionnés}
\label{sec:orgheadline11}
\begin{center}
\begin{tabular}{ll}
\toprule
strong & weak\\
\midrule
pS10 & pW14\\
pS24 & pW19\\
\bottomrule
\end{tabular}
\end{center}

\subsubsection{Programme PCR}
\label{sec:orgheadline12}
\begin{center}
\begin{tabular}{lrl}
\toprule
Temps & température & 30 cycles\\
\midrule
3' & 94 & \\
50'' & 94 & x\\
50'' & 55 & x\\
60'' & 72 & x\\
4' & 72 & \\
\bottomrule
\end{tabular}
\end{center}

\subsubsection{Volumes :}
\label{sec:orgheadline13}
\begin{center}
\begin{tabular}{lrr}
\toprule
 & 1-tube & 7-tubes\\
\midrule
ADN & 0.5 & 3.5\\
Taq & 0.5 & 3.5\\
amorce 1410 & 2.5 & 17.5\\
amorce 1411 & 2.5 & 17.5\\
MgCl2 & 1.5 & 10.5\\
dNTPs & 5.0 & 35.\\
Tampon 10X Taq & 5.0 & 35.\\
\midrule
eau qsp 50µL & 32.5 & 227.5\\
\bottomrule
\end{tabular}
\end{center}

\subsubsection{Contrôle :}
\label{sec:orgheadline14}

\begin{figure}[htb]
\centering
\includegraphics[width=4cm]{/Volumes/HDD/stage/img/20160113_00.JPG}
\caption{\label{fig:orgparagraph1}
Le contrôle de la PCR cible de l'ancre 100\% homologue. Le témoin positif de la PCR d'hier est censé être là également. }
\end{figure}

Voir figure \ref{fig:orgparagraph1}. 

\begin{itemize}
\item $\square$ Refaire migrer le témoin positif.
\end{itemize}

\subsection{(D) dilué amorces}
\label{sec:orgheadline16}
Ajouté le \textit{[2016-01-12 Tue 10:02]}

Dilué au 10\(^{\text{e}}\) (10 / 90µL eau aguettant) les amorces 1073, 1392, 1410 et 1411.

\subsection{(D) coulé gel}
\label{sec:orgheadline17}
Ajouté le \textit{[2016-01-12 Tue 16:01]}

Coulé deux grands gels 300mL agarose 1\% 26 puits.
\section{2016-01-13 Mercredi}
\label{sec:orgheadline29}
\subsection{(2) PCR cible des ancres}
\label{sec:orgheadline23}
Ajouté le \textit{[2016-01-13 Wed 09:36]}

La PCR d'hier ayant fonctionné, le but est de faire la même PCR dans les mêmes
conditions pour séquencer toutes les ancres des clones montrant des
néomutations. 

\subsubsection{candidats sélectionnés}
\label{sec:orgheadline19}
\begin{center}
\begin{tabular}{ll}
\toprule
strong & weak\\
\midrule
pS10 & pW14\\
pS24 & pW19\\
pS30 & pW2\\
pS39 & pW23\\
pS5 & pW35\\
pS54 & pW6\\
pS74 & pW81\\
pS82 & pW87\\
pS88 & pW93\\
\bottomrule
\end{tabular}
\end{center}

\begin{itemize}
\item erreur : trompé de clone, prélevé pW5 au lieu de pW2… gros malin.
\end{itemize}

\subsubsection{programme PCR}
\label{sec:orgheadline20}
\begin{center}
\begin{tabular}{lrl}
\toprule
Temps & température & 30 cycles\\
\midrule
3' & 94 & \\
50'' & 94 & x\\
50'' & 55 & x\\
60'' & 72 & x\\
4' & 72 & \\
\bottomrule
\end{tabular}
\end{center}

\subsubsection{volumes}
\label{sec:orgheadline21}
\begin{center}
\begin{tabular}{lrr}
\toprule
 & 1-tube & 19-tubes\\
\midrule
ADN & 0.5 & 9.5\\
Taq & 0.5 & 9.5\\
amorce 1410 & 2.5 & 47.5\\
amorce 1411 & 2.5 & 47.5\\
MgCl2 & 1.5 & 28.5\\
dNTPs & 5.0 & 95.\\
Tampon 10X Taq & 5.0 & 95.\\
\midrule
eau qsp 50µL & 32.5 & 617.5\\
\bottomrule
\end{tabular}
\end{center}

\subsubsection{contrôle}
\label{sec:orgheadline22}
Voir figure \ref{fig:orgparagraph2}. 

\begin{figure}[htb]
\centering
\includegraphics[width=\linewidth]{/Volumes/HDD/stage/img/20160113_01.JPG}
\caption{\label{fig:orgparagraph2}
La PCR cible des ancres semble avoir fonctionné de partout. Dans l'ordre, de pS10 à pS88, puis pW14 à pW93, T- et T+. Le T- est clean, le T+ trop fort. Penser à moins déposer pour le T+.}
\end{figure}

Tout a fonctionné a priori. Les produits de PCR sont conservés au -20°C dans le
portoir orange à couvercle. Je refais la PCR pour l'isolat pW2.

\subsection{(3) PCR cible des gènes synthétiques}
\label{sec:orgheadline28}
Il n'y a pas de bandes dans trois cas d'amplification, l'hypothèse la plus
parcimonieuse étant que la PCR n'a pas fonctionné… Je la refais donc pour ces
trois isolats là. lala. 

\subsubsection{candidats}
\label{sec:orgheadline24}
pW5\(_{\text{3}}\) -- pW5\(_{\text{4}}\) -- pS53\(_{\text{3}}\) 

\subsubsection{programme PCR}
\label{sec:orgheadline25}
\begin{center}
\begin{tabular}{lrl}
\toprule
Temps & température & 30 cycles\\
\midrule
3' & 94 & \\
50'' & 94 & x\\
50'' & 58 & x\\
50'' & 72 & x\\
4' & 72 & \\
\bottomrule
\end{tabular}
\end{center}

\subsubsection{volumes}
\label{sec:orgheadline26}
\begin{center}
\begin{tabular}{lrr}
\toprule
 & 1-tube & 6-tubes\\
\midrule
ADN & 0.5 & 3.\\
Taq & 0.5 & 3.\\
amorce 1410 & 2.5 & 15.\\
amorce 1411 & 2.5 & 15.\\
MgCl2 & 1.5 & 9.\\
dNTPs & 5.0 & 30.\\
Tampon 10X Taq & 5.0 & 30.\\
\midrule
eau qsp 50µL & 32.5 & 195.\\
\bottomrule
\end{tabular}
\end{center}

\subsubsection{contrôle}
\label{sec:orgheadline27}
Voir figure \ref{fig:orgparagraph3}. 
Absence d'amplification. Repartir des boîtes, resuspendre les colonies, et
refaire la PCR. 

\begin{figure}[htb]
\centering
\includegraphics[width=0.3\linewidth]{/Volumes/HDD/stage/img/20160113_02.JPG}
\caption{\label{fig:orgparagraph3}
Rien à l'horizon. Ladder tordu, puisque gel qui se tortille dans la cuve… Bien fait pour moi.}
\end{figure}

\section{2016-01-14 Jeudi}
\label{sec:orgheadline38}
\subsection{(3) PCR des trois clones récalcitrants}
\label{sec:orgheadline34}
La PCR d'hier semble montrer qu'il n'y a pas d'ADN dans le tube. Le témoin
positif, qui fonctionne bien avec le couple d'amorce 1410--1411 ne fonctionne
pas avec les amorces 1073--1392, pour une raison que je ne m'explique encore
pas. 

Je veux donc repartir d'une nouvelle suspension, et refaire (encore une fois) la
PCR sur ces trois colonies. Le but est d'avoir un plan d'expérience équilibré,
avec quatre colonies isolées par candidat choisi.
\subsubsection{suspension des clones}
\label{sec:orgheadline30}
Les trois colonies sélectionnées, pW5\(_{\text{3}}\) -- pW5\(_{\text{4}}\) -- pS53\(_{\text{3}}\), sont resuspendues
dans 20µL d'eau pure. 
\subsubsection{programme PCR}
\label{sec:orgheadline31}
\begin{center}
\begin{tabular}{lrl}
\toprule
Temps & température & 30 cycles\\
\midrule
3' & 94 & \\
50'' & 94 & x\\
50'' & 58 & x\\
50'' & 72 & x\\
4' & 72 & \\
\bottomrule
\end{tabular}
\end{center}

\subsubsection{volumes}
\label{sec:orgheadline32}
\begin{center}
\begin{tabular}{lrr}
\toprule
 & 1-tube & 6-tubes\\
\midrule
ADN & 0.5 & 3.\\
Taq & 0.5 & 3.\\
amorce 1410 & 2.5 & 15.\\
amorce 1411 & 2.5 & 15.\\
MgCl2 & 1.5 & 9.\\
dNTPs & 5.0 & 30.\\
Tampon 10X Taq & 5.0 & 30.\\
\midrule
eau qsp 50µL & 32.5 & 195.\\
\bottomrule
\end{tabular}
\end{center}

\subsubsection{contrôle}
\label{sec:orgheadline33}

\begin{figure}[htb]
\centering
\includegraphics[width=0.5\linewidth]{/Volumes/HDD/stage/img/20160114_00.JPG}
\caption{Les trois clones ont bien amplifié. Le T+ a toujours le même soucis, je suis pourtant parti d'un nouvel ADN génomique d'Acineto.}
\end{figure}

\subsection{(1) Contrôle des extractions plasmidiques}
\label{sec:orgheadline36}
Les extractions des plasmides porteurs des gènes synthétiques du
\textit{[2016-01-08 Fri] } sont contrôlées au nano-drop.

Résultat :
\begin{center}
\begin{tabular}{ll}
\toprule
plasmide & dosage\\
\midrule
WS & 8.5ng/µL\\
SW & 7.2ng/µL\\
\bottomrule
\end{tabular}
\end{center}

Il n'y a pratiquement rien. 

\subsubsection{relance des cultures}
\label{sec:orgheadline35}
J'ai sorti les cryotubes du -80°C, resuspendu les cultures dans 5mL de LB+Kan50,
incubation 24H à 37°C. Le but est de refaire les extractions demain
\textit{[2016-01-15 Fri] } avec plus de milieu, en culottant les 5mL de culture, et en
éluant dans un plus petit volume.

\subsection{(D) maintenance}
\label{sec:orgheadline37}
\begin{itemize}
\item nettoyé paillasse
\item aliquoté 10 \texttimes{} 1.5mL d'eau UP en eppendorf.
\end{itemize}
\section{2016-01-15 Vendredi}
\label{sec:orgheadline44}
\subsection{(1) Extractions plasmidiques}
\label{sec:orgheadline42}
\subsubsection{Extractions}
\label{sec:orgheadline39}
Vu les dosages nanodrop des extractions précédentes, je refais l'extraction sur
une culture liquide fraîche. Culotté 5mL et suivi le protocole standard. Élution
en deux étapes (2\texttimes{}15µL), incubation 3min à °C ambiante. Les tubes sont
appelés \texttt{WS} et \texttt{SW}, \textit{[2016-01-15 Fri]}.

\subsubsection{Contrôle}
\label{sec:orgheadline40}

\begin{figure}[htb]
\centering
\includegraphics[width=0.25\linewidth]{/Volumes/HDD/stage/img/20160115_00.JPG}
\caption{Le plasmide WS semble bon, il ne smirre pas trop. Le plasmide SW est à refaire. }
\end{figure}

\subsubsection{NanoDrop}
\label{sec:orgheadline41}
\begin{center}
\begin{tabular}{lr}
\toprule
 & dosage (ng/µL)\\
\midrule
WS & 25.2\\
 & 25.8\\
\midrule
SW & 6.7\\
 & 5.9\\
 & 9.1\\
 & 5.0\\
 & nul quoi. 0\\
\bottomrule
\end{tabular}
\end{center}
\subsection{(A) Analyses}
\label{sec:orgheadline43}
Avancé sur les alignements polySNP, sur l'extraction des données comme vu avec
Laurent hier \textit{[2016-01-14 Thu]}. 
\section{2016-01-18 Lundi}
\label{sec:orgheadline52}
\subsection{(1) Constructions In-Fusion}
\label{sec:orgheadline45}
On veut mettre au point une méthode de clonage par PCR qui permettrait d'accoler
les fragments GS (gène synthétique) avec apha3 (kanamycine) et l'ancre.

Les kits In-Fusion permettent de concevoir ça in-silico.

Le vecteur utilisé est le plasmide pGEMT-T. Il est ouvert par digestion SpeI. Il
faut absence de site de restriction SpeI dans le GS. Ç'est le cas. Vérifié le
\textit{<2016-01-18 Mon 08:54>}.

\begin{figure}[htb]
\centering
\includegraphics[width=\linewidth]{/Volumes/HDD/stage/img/20160118_00.png}
\caption{Fragment 1 : le Gene Synthétique. Fragment 2 : le gène aphA-3. Fragment 2 : le gène ancre.}
\end{figure}
\subsection{(2) PCR pW2 avant sequençage}
\label{sec:orgheadline48}
L'isolât pW2 n'a pas été amplifié, confondu les tubes (voir figure
\ref{fig:orgparagraph2}). On refait la PCR cible ancre sur cet isolat là. 

\subsubsection{programme PCR}
\label{sec:orgheadline46}
\begin{center}
\begin{tabular}{lrl}
\toprule
Temps & température & 30 cycles\\
\midrule
3' & 94 & \\
50'' & 94 & x\\
50'' & 55 & x\\
60'' & 72 & x\\
4' & 72 & \\
\bottomrule
\end{tabular}
\end{center}

\subsubsection{volumes}
\label{sec:orgheadline47}
\begin{center}
\begin{tabular}{lrr}
\toprule
 & 1-tube & 4-tubes\\
\midrule
ADN & 0.5 & 2.\\
Taq & 0.5 & 2.\\
amorce 1410 & 2.5 & 10.\\
amorce 1411 & 2.5 & 10.\\
MgCl2 & 1.5 & 6.\\
dNTPs & 5.0 & 20.\\
Tampon 10X Taq & 5.0 & 20.\\
\midrule
eau qsp 50µL & 32.5 & 130.\\
\bottomrule
\end{tabular}
\end{center}

\subsection{(2) Envoi ancres sequençage.}
\label{sec:orgheadline49}
Les produits PCR du \textit{[2016-01-13 Wed] } sont envoyés à séquencer. But : séquencer
les ancres des clones qui montrent des néomutations. 
\subsection{(3) Envoi produits PCR sequencage.}
\label{sec:orgheadline50}
Les produits PCR du \textit{[2016-01-12 Tue] } sont envoyés à séquencer. But : séquencer
les gènes synthétiques de certains candidats qui montrent des polymorphismes
assez marqués. 
\subsection{(D) Aliquoté amorces}
\label{sec:orgheadline51}
Aliquoté 3\texttimes{} 100mL des amorces amorces \texttt{1073 1392 1411 1410} en diluant au
10\(^{\text{e}}\).
\section{2016-01-19 Mardi}
\label{sec:orgheadline58}
\subsection{(1) PCR cible GS}
\label{sec:orgheadline56}
But : insérer les gènes synthétiques alternants dans pGEM-T, après une PCR \texttt{A}
sortant. 
\subsubsection{Programme PCR :}
\label{sec:orgheadline53}
\begin{center}
\begin{tabular}{lrl}
\toprule
Temps & température & 30 cycles\\
\midrule
3' & 94 & \\
50'' & 94 & x\\
50'' & 58 & x\\
50'' & 72 & x\\
4' & 72 & \\
\bottomrule
\end{tabular}
\end{center}
\subsubsection{Volumes :}
\label{sec:orgheadline54}
\begin{center}
\begin{tabular}{lrr}
\toprule
 & 1-tube & 7-tubes\\
\midrule
ADN & 0.5 & 3.5\\
Taq & 0.5 & 3.5\\
amorce 1392 & 2.5 & 17.5\\
amorce 1073 & 2.5 & 17.5\\
MgCl2 & 1.5 & 10.5\\
dNTPs & 5.0 & 35.\\
Tampon 10X Taq & 5.0 & 35.\\
\midrule
eau qsp 50µL & 32.5 & 227.5\\
\bottomrule
\end{tabular}
\end{center}

Problème : pas le bon couple d'amorce. 1073 est une amorce spécifique \emph{Acineto}.
Le couple d'amorce nécessaire est le couple \texttt{1392 1393}. La PCR est refaite
demain \textit{<2016-01-20 Wed 07:00>}.

\subsubsection{contrôle}
\label{sec:orgheadline55}
Comme attendu, pas de bandes. En plus la PCR est un peu sale. Penser à changer
l'eau. 


\begin{figure}[htb]
\centering
\includegraphics[width=3cm]{/Volumes/HDD/stage/img/20160119_00.JPG}
\caption{ }
\end{figure}


\subsection{(1) Couler boîtes Amp75 XGal IPTG}
\label{sec:orgheadline57}
Ces boîtes servent à sélectionner les transformants qui possèdent le gène
synthétique dans le site de clonage. La disruption du gène de la \(\beta\)-gal par
l'insertion donne des colonies blanches résistantes à l'ampicilline.

Couler 10 boîtes de LBm aux concentrations :
\begin{center}
\begin{tabular}{lll}
\toprule
 & [] & V\(_{\text{0}}\)\\
\midrule
Xgal & 60µg/mL & 115 µL\\
IPTG & 40µg/mL & 0.5 mL\\
Amp75 & 75µg/mL & 0.25mL\\
\bottomrule
\end{tabular}
\end{center}

\section{2016-01-20 Mercredi}
\label{sec:orgheadline69}
\subsection{(1) PCR cible GS}
\label{sec:orgheadline62}
Le but est d'amplifier les fragments synthétiques, dans l'idée de les insérer
après purification dans le plasmide pGEM-T. La PCR donne des \texttt{A} sortants,
ligaturés dans les \texttt{T} sortant du plasmide pGEM-T. 

\subsubsection{Programme PCR :}
\label{sec:orgheadline59}
D'après le cahier de Florence, daté du \textit{[2015-05-15 Fri]}, page 120. 
\begin{center}
\begin{tabular}{lrl}
\toprule
Temps & température & 30 cycles\\
\midrule
3' & 94 & \\
50'' & 94 & x\\
50'' & 60 & x\\
45'' & 72 & x\\
4' & 72 & \\
\bottomrule
\end{tabular}
\end{center}
\subsubsection{Volumes :}
\label{sec:orgheadline60}
\begin{center}
\begin{tabular}{lrr}
\toprule
 & 1-tube & 7-tubes\\
\midrule
ADN & 0.5 & 3.5\\
Taq & 0.5 & 3.5\\
amorce 1392 & 2.5 & 17.5\\
amorce 1393 & 2.5 & 17.5\\
MgCl2 & 1.5 & 10.5\\
dNTPs & 5.0 & 35.\\
Tampon 10X Taq & 5.0 & 35.\\
\midrule
eau qsp 50µL & 32.5 & 227.5\\
\bottomrule
\end{tabular}
\end{center}

Le T+ utilisé est un plasmide pGEM-T porteur de la construction Weak, synthétisé
par Florence. 
\subsubsection{contrôle}
\label{sec:orgheadline61}
\begin{figure}[htb]
\centering
\includegraphics[width=0.5\linewidth]{/Volumes/HDD/stage/img/20160120_00.JPG}
\caption{Contrôle de la PCR 1392-1393. Le témoin positif est un pGEM-T de Florence porteur de la construction Weak. }
\end{figure}

Tout est good. On peut lancer la purification, et éluer dans un bon volume de
25µL. 
\subsection{(1) Purification des fragments PCR amplifiés}
\label{sec:orgheadline64}
Les fragments PCR sont purifiés par le protocole
\href{///Users/samuelbarreto/Dropbox/Cours/Master/Semestre4/StageM2/doc/nucleospin_pcr_purif_cleanup.pdf::17}{nucleospin\(_{\text{pcr}}_{\text{purif}}_{\text{cleanup.pdf}}\), p. 17}. 

Les deux produits de PCR SW et WS sont poolés avant d'être mélangé au tampon de
charge de l'étape 1. L'éthanol à l'étape 4 est laissé à évaporer pendant 5min à
60°C, colonne ouverte dans le bloc chauffant. L'ADN est élué par un volume de
25µL par de l'eau UP chauffée préalablement à 60°C.

\subsubsection{Contrôle}
\label{sec:orgheadline63}

\begin{figure}[htb]
\centering
\includegraphics[width=0.3\linewidth]{/Volumes/HDD/stage/img/20160120_01.JPG}
\caption{Contrôle de la purification des produits PCR. Les deux purifications semblent avoir fonctionné comme il faut. 2µL sont chargés. 3µL de ladder ne suffisent pas. Il faut homogénéiser le ladder et charger plus.}
\end{figure}
\subsection{(1) Ligature dans pGEM-T}
\label{sec:orgheadline65}
Le plasmide pGEM-T de Promega est un plasmide conçu pour avoir des extrémités
franches porteuses de bases \texttt{T}. Il est maintenu ouvert dans son tampon,
conservé à -20°C. 

Le but est de ligaturer les fragments PCR obtenus et purifiés avec le pGEM-T,
puis de transformer des cellules compétentes \emph{E.coli} par choc chaud-froid. 

Le protocole utilisé est le protocole promega pGEM-T (pas pGEM-T easy). Lien
pour le short-manual : \href{///Users/samuelbarreto/Dropbox/Cours/Master/Semestre4/StageM2/doc/promega_pgem-t_short-manual.pdf::1}{promega\(_{\text{pgem}}\)-t\(_{\text{short}}\)-manual.pdf, p. 1}. Lien pour le
protocole complet : \href{///Users/samuelbarreto/Dropbox/Cours/Master/Semestre4/StageM2/doc/promega_pgem-t.pdf::5}{promega\(_{\text{pgem}}\)-t.pdf, p. 5}.

\subsection{(1) Transformation dans \emph{E.coli} TOP10}
\label{sec:orgheadline66}
Les produits de ligature sont utilisés pour transformer des \emph{E.coli} TOP10
thermocompétentes à la transformation. Le reste des produits de ligature est
placé au 4°C sur la nuit, pour favoriser l'apparition de transformants, en cas
de soucis.

Les transformants sont placés dans 300µL de milieu SOC, qui sont en étalés en 3
\texttimes{} 100µL sur les boîtes Amp75 XGal IPTG coulées hier \textit{[2016-01-19 Tue]}.

Les boîtes sont incubées 24H à 37°C. 

\subsection{(2) et (3) envoi au séquençage}
\label{sec:orgheadline67}
Les tubes à envoyer à séquencer, contenant les produits PCR cibles des gènes
synthétiques des clones présentant des polymorphismes et des ancres des clones
présentant des néomutations ont été envoyés à séquencer par David aujourd'hui
\textit{[2016-01-21 Thu]}. 

\subsection{(D) coulé boîtes amp-xgal-iptg}
\label{sec:orgheadline68}
Coulé 10 boîtes ampicilline 75, Xgal IPTG, placées en chambre froide.
\section{2016-01-21 Jeudi}
\label{sec:orgheadline73}
\subsection{(1) Contrôle des clonages}
\label{sec:orgheadline70}
Pas de culture. Aucun clone, ni bleu ni blanc. Thibault pense à un soucis avec
les cellules. Il pense aussi à la Taq ozyme, qui pourrait ne pas introduire de
\texttt{A} sortant. Mais dans ce cas, on aurait quand même des cellules bleues. Le
clonage de Raphaël ne fonctionne pas mieux. Le plus étonnant est qu'il n'y ait
pas plus de colonies bleues. Même si le clonage n'avait pas fonctionné, on
aurait quand même des cellules bleues.

\subsection{(1) Relance des clonages}
\label{sec:orgheadline71}
Les cellules qu'on avait utilisé hier sont périmées depuis septembre 2011. Ce
qui est très chiant. On refait donc le clonage avec le restant de produit de
ligature, conservé au 4°C depuis hier, en utilisant d'autres cellules, qui
périment en 2017. 

On utilise le kit de clonage TOP10. Voir description des protocoles ici :
\href{///Users/samuelbarreto/Dropbox/Cours/Master/Semestre4/StageM2/doc/invitrogen_one-shot-top10.pdf::8}{invitrogen\(_{\text{one}}\)-shot-top10.pdf, p. 8}.

Les transformants sont placés dans 300µL de milieu SOC, qui sont en étalés en 3
\texttimes{} 100µL sur les boîtes Amp75 XGal IPTG coulées hier \textit{[2016-01-20 Wed]}.

Les boîtes sont incubées 24H à 37°C. 

\subsection{(2) et (3) accusé réception séquençage}
\label{sec:orgheadline72}
\href{msgid:001501d1542e$943783b0$bca68b10$@genoscreen.fr}{GENOSCREEN : Réception des échantillons envoyés le 20/01/2016}

Genoscreen a reçu les échantillons. 

\section{2016-01-22 Vendredi}
\label{sec:orgheadline76}
\subsection{(1) Contrôle des clonages}
\label{sec:orgheadline74}
Ajouté le \textit{[2016-01-22 Fri 07:43] } \\
Beaucoup de colonies blanches et quelques colonies bleues. Le clonage d'hier
semble avoir fonctionné. On veut désormais repiquer les colonies blanches,
potentiellement transformantes, sur une autre boîte LBm Amp75 Xgal IPTG, 1) pour
confirmer leur phénotype, et 2) pour lancer des PCR de confirmation d'insertion
du fragment demain \textit{[2016-01-23 Sat]}. 

\subsection{(2) et (3) Réception des séquences}
\label{sec:orgheadline75}
Les séquences des tubes envoyés à Genoscreen ont été reçues aujourd'hui
\textit{<2016-01-22 Fri>}. 

\href{msgid:008901d15517$8e3c72b0$aab55810$@genoscreen.fr}{GENOSCREEN : Résultats des séquences du 20/01/2016 + rapport QC ( Mail 1/2)}
\href{msgid:008f01d15517$9b813500$d2839f00$@genoscreen.fr}{GENOSCREEN : Résultats des séquences du 20/01/2016 + rapport QC ( Mail 2/2)}
\section{2016-01-23 Samedi}
\label{sec:orgheadline81}
\subsection{(1) PCR contrôle des clonages}
\label{sec:orgheadline80}
Le but de cette PCR est de vérifier le sens des insertions des fragments de GS
dans pGEM-T. Avec le couple d'amorce V2 - M13R (1393 - M13R), on s'attend à
obtenir une taille d'amplification de 800bp environ, si le fragment est dans le
bon sens.

On choisit donc 6 clones confirmés \(\beta\)-gal neg par insert WS et SW. Ils sont
resuspendus dans 50µL d'eau stérile. 

\subsubsection{programme PCR}
\label{sec:orgheadline77}
Tiré du cahier de Florence, daté du \textit{[2015-06-16 Tue]}, page 135. 
\begin{center}
\begin{tabular}{lrl}
\toprule
Temps & température & 30 cycles\\
\midrule
3' & 94 & \\
50'' & 94 & x\\
50'' & 57 & x\\
50'' & 72 & x\\
4' & 72 & \\
\bottomrule
\end{tabular}
\end{center}

\subsubsection{volumes}
\label{sec:orgheadline78}
\begin{center}
\begin{tabular}{lrr}
\toprule
 & 1-tube & 14-tubes\\
\midrule
ADN & 0.5 & 7.\\
Taq & 0.5 & 7.\\
amorce M13R & 2.5 & 35.\\
amorce 1393 & 2.5 & 35.\\
MgCl2 & 1.5 & 21.\\
dNTPs & 5.0 & 70.\\
Tampon 10X Taq & 5.0 & 70.\\
\midrule
eau qsp 50µL & 32.5 & 455.\\
\bottomrule
\end{tabular}
\end{center}

\subsubsection{contrôles}
\label{sec:orgheadline79}
\begin{figure}[htb]
\centering
\includegraphics[width=\linewidth]{/Volumes/HDD/stage/img/20160123_00.JPG}
\caption{PCR 1}
\end{figure}

\begin{figure}[htb]
\centering
\includegraphics[width=\linewidth]{/Volumes/HDD/stage/img/20160123_01.JPG}
\caption{PCR 2}
\end{figure}


Rien à l'horizon. Pas même dans le témoin positif. J'ai refais la même PCR, en
passant d'un volume de 25µL à 50µL. Pas plus de chances.

Je n'ai pour l'instant pas d'explications.
\begin{enumerate}
\item les amorces sont nazes ? elles proviennent toutes les deux d'une dilution de
la solution mère, diluée ce matin pour la M13R, et mercredi \textit{[2016-01-20 Wed]}
pour la 1393.
\item la taq est naze ? J'ai essayé avec deux tubes différents.
\item le témoin positif est naze (l'insert dans ce tube n'est pas dans le bon sens)
\textbf{et} les 12 clones sont tout aussi naze ? allons allons, les lois de
l'échantillonnage ne sont quand même pas à ce point en ma défaveur…
\item ça n'est pas le bon couple d'amorce ? pourtant avec florence ça paraît pas
mal.
\item le temps d'élongation n'est pas le bon ? c'est pourtant le temps
correspondant à la taille attendue…
\end{enumerate}

Dégouté. Comment perdre son samedi en deux PCR. 


\section{2016-01-24 Dimanche}
\label{sec:orgheadline85}
\subsection{(1) Extraction Plasmidiques}
\label{sec:orgheadline82}
\subsection{(1) Ensemencements glycérols}
\label{sec:orgheadline83}


\subsection{(1) Ensemencement des tubes de LBm Amp75}
\label{sec:orgheadline84}
Deux clones par insert servent à ensemencer deux tubes de LBm Amp75, dans le but
de 1) faire les extractions plasmidiques demain \textit{<2016-01-24 Sun> } et 2) culotter
dans du glycérol, pour mettre au -80°C. 
\end{document}
               
\usepackage[utf8]{inputenc}
\usepackage[T1]{fontenc}
\usepackage{graphicx}
\usepackage{longtable}
\usepackage{float}
\usepackage{hyperref}
\usepackage{wrapfig}
\usepackage{rotating}
\usepackage[normalem]{ulem}
\usepackage{amsmath}
\usepackage{textcomp}
\usepackage{marvosym}
\usepackage{wasysym}
\usepackage{amssymb}
\usepackage[scaled=0.9]{zi4}
\usepackage[x11names, dvipsnames]{xcolor}
\usepackage[protrusion=true, expansion=alltext, tracking=true, kerning=true]{microtype}
\usepackage{siunitx}
\usepackage[french, frenchb]{babel}
\author{Samuel BARRETO}
\date{\today}
\title{}
\hypersetup{
 pdfauthor={Samuel BARRETO},
 pdftitle={},
 pdfkeywords={},
 pdfsubject={},
 pdfcreator={Emacs 24.5.1 (Org mode 8.3.3)}, 
 pdflang={Frenchb}}
\begin{document}

\setcounter{tocdepth}{2}
\tableofcontents

\clearpage

\part{2016-01 Janvier}
\label{sec:orgheadline86}
\section{2016-01-11 Lundi}
\label{sec:orgheadline6}
\subsection{(3) PCR cible des gènes synthétiques}
\label{sec:orgheadline4}
Ajouté le \textit{[2016-01-11 Mon 09:50]}

La PCR du \textit{[2016-01-07 Thu]}, cible des gènes synthétiques des clones
transformants hétérozygotes, n'a pas fonctionné, vraisemblablement dû à un
problème dans le mix PCR (pas d'amplification dans le t+). On refait donc la
même PCR sur moins de clones transformants, dans l'idée de se faire la main,
avant de faire des PCR plus conséquentes, avec de plus grands volumes. 

\subsubsection{Programme PCR :}
\label{sec:orgheadline1}
\begin{center}
\begin{tabular}{lrl}
\toprule
Temps & température & 30 cycles\\
\midrule
3' & 94 & \\
50'' & 94 & x\\
50'' & 58 & x\\
4' & 72 & x\\
\bottomrule
\end{tabular}
\end{center}

\subsubsection{Volumes :}
\label{sec:orgheadline2}
\begin{center}
\begin{tabular}{lrr}
\toprule
 & 1-tube & 7-tubes\\
\midrule
ADN & 0.5 & 3.5\\
Taq & 0.5 & 3.5\\
amorce 1392 & 2.5 & 17.5\\
amorce 1073 & 2.5 & 17.5\\
MgCl2 & 1.5 & 10.5\\
dNTPs & 5.0 & 35.\\
Tampon 10X Taq & 5.0 & 35.\\
\midrule
eau qsp 50µL & 32.5 & 227.5\\
\bottomrule
\end{tabular}
\end{center}

\subsubsection{Contrôle}
\label{sec:orgheadline3}
\begin{figure}[htb]
\centering
\includegraphics[width=0.4\linewidth]{/Volumes/HDD/stage/img/20160111_0.JPG}
\end{figure}

Ahah, la migration a duré trop longtemps à priori… Le bleu était sorti, petite
réunion avec Franck non-prévue…

On voit quand même des bandes dans les quatre puits, mais pas dans le témoin
positif, avec ADN génomique d'\emph{Acinetobacter}. Les bandes sont à la position
attendue dans le gel (taille : 750bp.) 

\begin{itemize}
\item voir à changer le témoin positif.
\end{itemize}

\subsection{(D) aliquoté Taq-ozyme}
\label{sec:orgheadline5}
Ajouté le \textit{[2016-01-11 Mon 11:47]}

Aliquoté 10µL de Taq-ozyme en eppendorf, placé dans le tiroir commun du -20°C.

\section{2016-01-12 Mardi}
\label{sec:orgheadline18}
\subsection{(3) PCR cible des gènes synthétiques}
\label{sec:orgheadline10}
Ajouté le \textit{[2016-01-12 Tue 08:19]}

La PCR d'hier ayant fonctionné, le but est de refaire la même PCR sur l'ensemble
des clones montrant des polymorphismes que j'ai isolé, avant de pouvoir les
envoyer à séquencer.

\begin{itemize}
\item Penser à changer de témoin positif (peut-être utiliser un plasmide porteur de
l'une des constructions par exemple.)
\end{itemize}

\subsubsection{Programme PCR :}
\label{sec:orgheadline7}
\begin{center}
\begin{tabular}{lrl}
\toprule
Temps & température & 30 cycles\\
\midrule
3' & 94 & \\
50'' & 94 & x\\
50'' & 58 & x\\
50'' & 72 & x\\
4' & 72 & \\
\bottomrule
\end{tabular}
\end{center}

\subsubsection{Volumes :}
\label{sec:orgheadline8}
\begin{center}
\begin{tabular}{lrr}
\toprule
 & 1-tube & 27-tubes\\
\midrule
ADN & 0.5 & 13.5\\
Taq & 0.5 & 13.5\\
amorce 1392 & 2.5 & 67.5\\
amorce 1073 & 2.5 & 67.5\\
MgCl2 & 1.5 & 40.5\\
dNTPs & 5.0 & 135.\\
Tampon 10X Taq & 5.0 & 135.\\
\midrule
eau qsp 50µL & 32.5 & 877.5\\
\bottomrule
\end{tabular}
\end{center}

\subsubsection{Contrôle :}
\label{sec:orgheadline9}

\begin{figure}[htb]
\centering
\includegraphics[width=3cm]{/Volumes/HDD/stage/img/20160112_00.JPG}
\caption{Ladder un peu chargé. Attention au volume. Colonies 3 et 4 de pW5 et 3 de pS53 n'ont pas amplifié. Témoin positif migration sur le gel de \textit{[2016-01-12 Tue 16:05]}.}
\end{figure}

\begin{itemize}
\item $\square$ refaire la PCR pour les deux clones pW5 et le clone de pS53 qui n'a pas
fonctionné.
\item $\square$ faire migrer le témoin positif.
\end{itemize}

\subsection{(2) PCR cible de l'ancre}
\label{sec:orgheadline15}
Ajouté le \textit{[2016-01-12 Tue 14:15]}

Chez les différents clones montrant des néomutations, on veut séquencer l'ancre,
pour vérifier que le taux de mutation dans la tract de conversion est bien
supérieur au taux de mutation dans une zone parfaitement homologue. L'hypothèse,
floue pour l'instant, est que le système de réparation des mésappariemments est
"saturé", et qu'il introduit des erreurs. (Ces mutations semblent biaisées vers
GC d'après les données dont on dispose pour l'instant.)

Le but est ici de réaliser une PCR simple sur quelques candidats, avant de
passer aux grands volumes, envoyés à séquencer. Les conditions de PCR sont
tirées du cahier de Florence, page 146, daté du \textit{[2015-06-25 Thu]}. 

\subsubsection{Candidats sélectionnés}
\label{sec:orgheadline11}
\begin{center}
\begin{tabular}{ll}
\toprule
strong & weak\\
\midrule
pS10 & pW14\\
pS24 & pW19\\
\bottomrule
\end{tabular}
\end{center}

\subsubsection{Programme PCR}
\label{sec:orgheadline12}
\begin{center}
\begin{tabular}{lrl}
\toprule
Temps & température & 30 cycles\\
\midrule
3' & 94 & \\
50'' & 94 & x\\
50'' & 55 & x\\
60'' & 72 & x\\
4' & 72 & \\
\bottomrule
\end{tabular}
\end{center}

\subsubsection{Volumes :}
\label{sec:orgheadline13}
\begin{center}
\begin{tabular}{lrr}
\toprule
 & 1-tube & 7-tubes\\
\midrule
ADN & 0.5 & 3.5\\
Taq & 0.5 & 3.5\\
amorce 1410 & 2.5 & 17.5\\
amorce 1411 & 2.5 & 17.5\\
MgCl2 & 1.5 & 10.5\\
dNTPs & 5.0 & 35.\\
Tampon 10X Taq & 5.0 & 35.\\
\midrule
eau qsp 50µL & 32.5 & 227.5\\
\bottomrule
\end{tabular}
\end{center}

\subsubsection{Contrôle :}
\label{sec:orgheadline14}

\begin{figure}[htb]
\centering
\includegraphics[width=4cm]{/Volumes/HDD/stage/img/20160113_00.JPG}
\caption{\label{fig:orgparagraph1}
Le contrôle de la PCR cible de l'ancre 100\% homologue. Le témoin positif de la PCR d'hier est censé être là également. }
\end{figure}

Voir figure \ref{fig:orgparagraph1}. 

\begin{itemize}
\item $\square$ Refaire migrer le témoin positif.
\end{itemize}

\subsection{(D) dilué amorces}
\label{sec:orgheadline16}
Ajouté le \textit{[2016-01-12 Tue 10:02]}

Dilué au 10\(^{\text{e}}\) (10 / 90µL eau aguettant) les amorces 1073, 1392, 1410 et 1411.

\subsection{(D) coulé gel}
\label{sec:orgheadline17}
Ajouté le \textit{[2016-01-12 Tue 16:01]}

Coulé deux grands gels 300mL agarose 1\% 26 puits.
\section{2016-01-13 Mercredi}
\label{sec:orgheadline29}
\subsection{(2) PCR cible des ancres}
\label{sec:orgheadline23}
Ajouté le \textit{[2016-01-13 Wed 09:36]}

La PCR d'hier ayant fonctionné, le but est de faire la même PCR dans les mêmes
conditions pour séquencer toutes les ancres des clones montrant des
néomutations. 

\subsubsection{candidats sélectionnés}
\label{sec:orgheadline19}
\begin{center}
\begin{tabular}{ll}
\toprule
strong & weak\\
\midrule
pS10 & pW14\\
pS24 & pW19\\
pS30 & pW2\\
pS39 & pW23\\
pS5 & pW35\\
pS54 & pW6\\
pS74 & pW81\\
pS82 & pW87\\
pS88 & pW93\\
\bottomrule
\end{tabular}
\end{center}

\begin{itemize}
\item erreur : trompé de clone, prélevé pW5 au lieu de pW2… gros malin.
\end{itemize}

\subsubsection{programme PCR}
\label{sec:orgheadline20}
\begin{center}
\begin{tabular}{lrl}
\toprule
Temps & température & 30 cycles\\
\midrule
3' & 94 & \\
50'' & 94 & x\\
50'' & 55 & x\\
60'' & 72 & x\\
4' & 72 & \\
\bottomrule
\end{tabular}
\end{center}

\subsubsection{volumes}
\label{sec:orgheadline21}
\begin{center}
\begin{tabular}{lrr}
\toprule
 & 1-tube & 19-tubes\\
\midrule
ADN & 0.5 & 9.5\\
Taq & 0.5 & 9.5\\
amorce 1410 & 2.5 & 47.5\\
amorce 1411 & 2.5 & 47.5\\
MgCl2 & 1.5 & 28.5\\
dNTPs & 5.0 & 95.\\
Tampon 10X Taq & 5.0 & 95.\\
\midrule
eau qsp 50µL & 32.5 & 617.5\\
\bottomrule
\end{tabular}
\end{center}

\subsubsection{contrôle}
\label{sec:orgheadline22}
Voir figure \ref{fig:orgparagraph2}. 

\begin{figure}[htb]
\centering
\includegraphics[width=\linewidth]{/Volumes/HDD/stage/img/20160113_01.JPG}
\caption{\label{fig:orgparagraph2}
La PCR cible des ancres semble avoir fonctionné de partout. Dans l'ordre, de pS10 à pS88, puis pW14 à pW93, T- et T+. Le T- est clean, le T+ trop fort. Penser à moins déposer pour le T+.}
\end{figure}

Tout a fonctionné a priori. Les produits de PCR sont conservés au -20°C dans le
portoir orange à couvercle. Je refais la PCR pour l'isolat pW2.

\subsection{(3) PCR cible des gènes synthétiques}
\label{sec:orgheadline28}
Il n'y a pas de bandes dans trois cas d'amplification, l'hypothèse la plus
parcimonieuse étant que la PCR n'a pas fonctionné… Je la refais donc pour ces
trois isolats là. lala. 

\subsubsection{candidats}
\label{sec:orgheadline24}
pW5\(_{\text{3}}\) -- pW5\(_{\text{4}}\) -- pS53\(_{\text{3}}\) 

\subsubsection{programme PCR}
\label{sec:orgheadline25}
\begin{center}
\begin{tabular}{lrl}
\toprule
Temps & température & 30 cycles\\
\midrule
3' & 94 & \\
50'' & 94 & x\\
50'' & 58 & x\\
50'' & 72 & x\\
4' & 72 & \\
\bottomrule
\end{tabular}
\end{center}

\subsubsection{volumes}
\label{sec:orgheadline26}
\begin{center}
\begin{tabular}{lrr}
\toprule
 & 1-tube & 6-tubes\\
\midrule
ADN & 0.5 & 3.\\
Taq & 0.5 & 3.\\
amorce 1410 & 2.5 & 15.\\
amorce 1411 & 2.5 & 15.\\
MgCl2 & 1.5 & 9.\\
dNTPs & 5.0 & 30.\\
Tampon 10X Taq & 5.0 & 30.\\
\midrule
eau qsp 50µL & 32.5 & 195.\\
\bottomrule
\end{tabular}
\end{center}

\subsubsection{contrôle}
\label{sec:orgheadline27}
Voir figure \ref{fig:orgparagraph3}. 
Absence d'amplification. Repartir des boîtes, resuspendre les colonies, et
refaire la PCR. 

\begin{figure}[htb]
\centering
\includegraphics[width=0.3\linewidth]{/Volumes/HDD/stage/img/20160113_02.JPG}
\caption{\label{fig:orgparagraph3}
Rien à l'horizon. Ladder tordu, puisque gel qui se tortille dans la cuve… Bien fait pour moi.}
\end{figure}

\section{2016-01-14 Jeudi}
\label{sec:orgheadline38}
\subsection{(3) PCR des trois clones récalcitrants}
\label{sec:orgheadline34}
La PCR d'hier semble montrer qu'il n'y a pas d'ADN dans le tube. Le témoin
positif, qui fonctionne bien avec le couple d'amorce 1410--1411 ne fonctionne
pas avec les amorces 1073--1392, pour une raison que je ne m'explique encore
pas. 

Je veux donc repartir d'une nouvelle suspension, et refaire (encore une fois) la
PCR sur ces trois colonies. Le but est d'avoir un plan d'expérience équilibré,
avec quatre colonies isolées par candidat choisi.
\subsubsection{suspension des clones}
\label{sec:orgheadline30}
Les trois colonies sélectionnées, pW5\(_{\text{3}}\) -- pW5\(_{\text{4}}\) -- pS53\(_{\text{3}}\), sont resuspendues
dans 20µL d'eau pure. 
\subsubsection{programme PCR}
\label{sec:orgheadline31}
\begin{center}
\begin{tabular}{lrl}
\toprule
Temps & température & 30 cycles\\
\midrule
3' & 94 & \\
50'' & 94 & x\\
50'' & 58 & x\\
50'' & 72 & x\\
4' & 72 & \\
\bottomrule
\end{tabular}
\end{center}

\subsubsection{volumes}
\label{sec:orgheadline32}
\begin{center}
\begin{tabular}{lrr}
\toprule
 & 1-tube & 6-tubes\\
\midrule
ADN & 0.5 & 3.\\
Taq & 0.5 & 3.\\
amorce 1410 & 2.5 & 15.\\
amorce 1411 & 2.5 & 15.\\
MgCl2 & 1.5 & 9.\\
dNTPs & 5.0 & 30.\\
Tampon 10X Taq & 5.0 & 30.\\
\midrule
eau qsp 50µL & 32.5 & 195.\\
\bottomrule
\end{tabular}
\end{center}

\subsubsection{contrôle}
\label{sec:orgheadline33}

\begin{figure}[htb]
\centering
\includegraphics[width=0.5\linewidth]{/Volumes/HDD/stage/img/20160114_00.JPG}
\caption{Les trois clones ont bien amplifié. Le T+ a toujours le même soucis, je suis pourtant parti d'un nouvel ADN génomique d'Acineto.}
\end{figure}

\subsection{(1) Contrôle des extractions plasmidiques}
\label{sec:orgheadline36}
Les extractions des plasmides porteurs des gènes synthétiques du
\textit{[2016-01-08 Fri] } sont contrôlées au nano-drop.

Résultat :
\begin{center}
\begin{tabular}{ll}
\toprule
plasmide & dosage\\
\midrule
WS & 8.5ng/µL\\
SW & 7.2ng/µL\\
\bottomrule
\end{tabular}
\end{center}

Il n'y a pratiquement rien. 

\subsubsection{relance des cultures}
\label{sec:orgheadline35}
J'ai sorti les cryotubes du -80°C, resuspendu les cultures dans 5mL de LB+Kan50,
incubation 24H à 37°C. Le but est de refaire les extractions demain
\textit{[2016-01-15 Fri] } avec plus de milieu, en culottant les 5mL de culture, et en
éluant dans un plus petit volume.

\subsection{(D) maintenance}
\label{sec:orgheadline37}
\begin{itemize}
\item nettoyé paillasse
\item aliquoté 10 \texttimes{} 1.5mL d'eau UP en eppendorf.
\end{itemize}
\section{2016-01-15 Vendredi}
\label{sec:orgheadline44}
\subsection{(1) Extractions plasmidiques}
\label{sec:orgheadline42}
\subsubsection{Extractions}
\label{sec:orgheadline39}
Vu les dosages nanodrop des extractions précédentes, je refais l'extraction sur
une culture liquide fraîche. Culotté 5mL et suivi le protocole standard. Élution
en deux étapes (2\texttimes{}15µL), incubation 3min à °C ambiante. Les tubes sont
appelés \texttt{WS} et \texttt{SW}, \textit{[2016-01-15 Fri]}.

\subsubsection{Contrôle}
\label{sec:orgheadline40}

\begin{figure}[htb]
\centering
\includegraphics[width=0.25\linewidth]{/Volumes/HDD/stage/img/20160115_00.JPG}
\caption{Le plasmide WS semble bon, il ne smirre pas trop. Le plasmide SW est à refaire. }
\end{figure}

\subsubsection{NanoDrop}
\label{sec:orgheadline41}
\begin{center}
\begin{tabular}{lr}
\toprule
 & dosage (ng/µL)\\
\midrule
WS & 25.2\\
 & 25.8\\
\midrule
SW & 6.7\\
 & 5.9\\
 & 9.1\\
 & 5.0\\
 & nul quoi. 0\\
\bottomrule
\end{tabular}
\end{center}
\subsection{(A) Analyses}
\label{sec:orgheadline43}
Avancé sur les alignements polySNP, sur l'extraction des données comme vu avec
Laurent hier \textit{[2016-01-14 Thu]}. 
\section{2016-01-18 Lundi}
\label{sec:orgheadline52}
\subsection{(1) Constructions In-Fusion}
\label{sec:orgheadline45}
On veut mettre au point une méthode de clonage par PCR qui permettrait d'accoler
les fragments GS (gène synthétique) avec apha3 (kanamycine) et l'ancre.

Les kits In-Fusion permettent de concevoir ça in-silico.

Le vecteur utilisé est le plasmide pGEMT-T. Il est ouvert par digestion SpeI. Il
faut absence de site de restriction SpeI dans le GS. Ç'est le cas. Vérifié le
\textit{<2016-01-18 Mon 08:54>}.

\begin{figure}[htb]
\centering
\includegraphics[width=\linewidth]{/Volumes/HDD/stage/img/20160118_00.png}
\caption{Fragment 1 : le Gene Synthétique. Fragment 2 : le gène aphA-3. Fragment 2 : le gène ancre.}
\end{figure}
\subsection{(2) PCR pW2 avant sequençage}
\label{sec:orgheadline48}
L'isolât pW2 n'a pas été amplifié, confondu les tubes (voir figure
\ref{fig:orgparagraph2}). On refait la PCR cible ancre sur cet isolat là. 

\subsubsection{programme PCR}
\label{sec:orgheadline46}
\begin{center}
\begin{tabular}{lrl}
\toprule
Temps & température & 30 cycles\\
\midrule
3' & 94 & \\
50'' & 94 & x\\
50'' & 55 & x\\
60'' & 72 & x\\
4' & 72 & \\
\bottomrule
\end{tabular}
\end{center}

\subsubsection{volumes}
\label{sec:orgheadline47}
\begin{center}
\begin{tabular}{lrr}
\toprule
 & 1-tube & 4-tubes\\
\midrule
ADN & 0.5 & 2.\\
Taq & 0.5 & 2.\\
amorce 1410 & 2.5 & 10.\\
amorce 1411 & 2.5 & 10.\\
MgCl2 & 1.5 & 6.\\
dNTPs & 5.0 & 20.\\
Tampon 10X Taq & 5.0 & 20.\\
\midrule
eau qsp 50µL & 32.5 & 130.\\
\bottomrule
\end{tabular}
\end{center}

\subsection{(2) Envoi ancres sequençage.}
\label{sec:orgheadline49}
Les produits PCR du \textit{[2016-01-13 Wed] } sont envoyés à séquencer. But : séquencer
les ancres des clones qui montrent des néomutations. 
\subsection{(3) Envoi produits PCR sequencage.}
\label{sec:orgheadline50}
Les produits PCR du \textit{[2016-01-12 Tue] } sont envoyés à séquencer. But : séquencer
les gènes synthétiques de certains candidats qui montrent des polymorphismes
assez marqués. 
\subsection{(D) Aliquoté amorces}
\label{sec:orgheadline51}
Aliquoté 3\texttimes{} 100mL des amorces amorces \texttt{1073 1392 1411 1410} en diluant au
10\(^{\text{e}}\).
\section{2016-01-19 Mardi}
\label{sec:orgheadline58}
\subsection{(1) PCR cible GS}
\label{sec:orgheadline56}
But : insérer les gènes synthétiques alternants dans pGEM-T, après une PCR \texttt{A}
sortant. 
\subsubsection{Programme PCR :}
\label{sec:orgheadline53}
\begin{center}
\begin{tabular}{lrl}
\toprule
Temps & température & 30 cycles\\
\midrule
3' & 94 & \\
50'' & 94 & x\\
50'' & 58 & x\\
50'' & 72 & x\\
4' & 72 & \\
\bottomrule
\end{tabular}
\end{center}
\subsubsection{Volumes :}
\label{sec:orgheadline54}
\begin{center}
\begin{tabular}{lrr}
\toprule
 & 1-tube & 7-tubes\\
\midrule
ADN & 0.5 & 3.5\\
Taq & 0.5 & 3.5\\
amorce 1392 & 2.5 & 17.5\\
amorce 1073 & 2.5 & 17.5\\
MgCl2 & 1.5 & 10.5\\
dNTPs & 5.0 & 35.\\
Tampon 10X Taq & 5.0 & 35.\\
\midrule
eau qsp 50µL & 32.5 & 227.5\\
\bottomrule
\end{tabular}
\end{center}

Problème : pas le bon couple d'amorce. 1073 est une amorce spécifique \emph{Acineto}.
Le couple d'amorce nécessaire est le couple \texttt{1392 1393}. La PCR est refaite
demain \textit{<2016-01-20 Wed 07:00>}.

\subsubsection{contrôle}
\label{sec:orgheadline55}
Comme attendu, pas de bandes. En plus la PCR est un peu sale. Penser à changer
l'eau. 


\begin{figure}[htb]
\centering
\includegraphics[width=3cm]{/Volumes/HDD/stage/img/20160119_00.JPG}
\caption{ }
\end{figure}


\subsection{(1) Couler boîtes Amp75 XGal IPTG}
\label{sec:orgheadline57}
Ces boîtes servent à sélectionner les transformants qui possèdent le gène
synthétique dans le site de clonage. La disruption du gène de la \(\beta\)-gal par
l'insertion donne des colonies blanches résistantes à l'ampicilline.

Couler 10 boîtes de LBm aux concentrations :
\begin{center}
\begin{tabular}{lll}
\toprule
 & [] & V\(_{\text{0}}\)\\
\midrule
Xgal & 60µg/mL & 115 µL\\
IPTG & 40µg/mL & 0.5 mL\\
Amp75 & 75µg/mL & 0.25mL\\
\bottomrule
\end{tabular}
\end{center}

\section{2016-01-20 Mercredi}
\label{sec:orgheadline69}
\subsection{(1) PCR cible GS}
\label{sec:orgheadline62}
Le but est d'amplifier les fragments synthétiques, dans l'idée de les insérer
après purification dans le plasmide pGEM-T. La PCR donne des \texttt{A} sortants,
ligaturés dans les \texttt{T} sortant du plasmide pGEM-T. 

\subsubsection{Programme PCR :}
\label{sec:orgheadline59}
D'après le cahier de Florence, daté du \textit{[2015-05-15 Fri]}, page 120. 
\begin{center}
\begin{tabular}{lrl}
\toprule
Temps & température & 30 cycles\\
\midrule
3' & 94 & \\
50'' & 94 & x\\
50'' & 60 & x\\
45'' & 72 & x\\
4' & 72 & \\
\bottomrule
\end{tabular}
\end{center}
\subsubsection{Volumes :}
\label{sec:orgheadline60}
\begin{center}
\begin{tabular}{lrr}
\toprule
 & 1-tube & 7-tubes\\
\midrule
ADN & 0.5 & 3.5\\
Taq & 0.5 & 3.5\\
amorce 1392 & 2.5 & 17.5\\
amorce 1393 & 2.5 & 17.5\\
MgCl2 & 1.5 & 10.5\\
dNTPs & 5.0 & 35.\\
Tampon 10X Taq & 5.0 & 35.\\
\midrule
eau qsp 50µL & 32.5 & 227.5\\
\bottomrule
\end{tabular}
\end{center}

Le T+ utilisé est un plasmide pGEM-T porteur de la construction Weak, synthétisé
par Florence. 
\subsubsection{contrôle}
\label{sec:orgheadline61}
\begin{figure}[htb]
\centering
\includegraphics[width=0.5\linewidth]{/Volumes/HDD/stage/img/20160120_00.JPG}
\caption{Contrôle de la PCR 1392-1393. Le témoin positif est un pGEM-T de Florence porteur de la construction Weak. }
\end{figure}

Tout est good. On peut lancer la purification, et éluer dans un bon volume de
25µL. 
\subsection{(1) Purification des fragments PCR amplifiés}
\label{sec:orgheadline64}
Les fragments PCR sont purifiés par le protocole
\href{///Users/samuelbarreto/Dropbox/Cours/Master/Semestre4/StageM2/doc/nucleospin_pcr_purif_cleanup.pdf::17}{nucleospin\(_{\text{pcr}}_{\text{purif}}_{\text{cleanup.pdf}}\), p. 17}. 

Les deux produits de PCR SW et WS sont poolés avant d'être mélangé au tampon de
charge de l'étape 1. L'éthanol à l'étape 4 est laissé à évaporer pendant 5min à
60°C, colonne ouverte dans le bloc chauffant. L'ADN est élué par un volume de
25µL par de l'eau UP chauffée préalablement à 60°C.

\subsubsection{Contrôle}
\label{sec:orgheadline63}

\begin{figure}[htb]
\centering
\includegraphics[width=0.3\linewidth]{/Volumes/HDD/stage/img/20160120_01.JPG}
\caption{Contrôle de la purification des produits PCR. Les deux purifications semblent avoir fonctionné comme il faut. 2µL sont chargés. 3µL de ladder ne suffisent pas. Il faut homogénéiser le ladder et charger plus.}
\end{figure}
\subsection{(1) Ligature dans pGEM-T}
\label{sec:orgheadline65}
Le plasmide pGEM-T de Promega est un plasmide conçu pour avoir des extrémités
franches porteuses de bases \texttt{T}. Il est maintenu ouvert dans son tampon,
conservé à -20°C. 

Le but est de ligaturer les fragments PCR obtenus et purifiés avec le pGEM-T,
puis de transformer des cellules compétentes \emph{E.coli} par choc chaud-froid. 

Le protocole utilisé est le protocole promega pGEM-T (pas pGEM-T easy). Lien
pour le short-manual : \href{///Users/samuelbarreto/Dropbox/Cours/Master/Semestre4/StageM2/doc/promega_pgem-t_short-manual.pdf::1}{promega\(_{\text{pgem}}\)-t\(_{\text{short}}\)-manual.pdf, p. 1}. Lien pour le
protocole complet : \href{///Users/samuelbarreto/Dropbox/Cours/Master/Semestre4/StageM2/doc/promega_pgem-t.pdf::5}{promega\(_{\text{pgem}}\)-t.pdf, p. 5}.

\subsection{(1) Transformation dans \emph{E.coli} TOP10}
\label{sec:orgheadline66}
Les produits de ligature sont utilisés pour transformer des \emph{E.coli} TOP10
thermocompétentes à la transformation. Le reste des produits de ligature est
placé au 4°C sur la nuit, pour favoriser l'apparition de transformants, en cas
de soucis.

Les transformants sont placés dans 300µL de milieu SOC, qui sont en étalés en 3
\texttimes{} 100µL sur les boîtes Amp75 XGal IPTG coulées hier \textit{[2016-01-19 Tue]}.

Les boîtes sont incubées 24H à 37°C. 

\subsection{(2) et (3) envoi au séquençage}
\label{sec:orgheadline67}
Les tubes à envoyer à séquencer, contenant les produits PCR cibles des gènes
synthétiques des clones présentant des polymorphismes et des ancres des clones
présentant des néomutations ont été envoyés à séquencer par David aujourd'hui
\textit{[2016-01-21 Thu]}. 

\subsection{(D) coulé boîtes amp-xgal-iptg}
\label{sec:orgheadline68}
Coulé 10 boîtes ampicilline 75, Xgal IPTG, placées en chambre froide.
\section{2016-01-21 Jeudi}
\label{sec:orgheadline73}
\subsection{(1) Contrôle des clonages}
\label{sec:orgheadline70}
Pas de culture. Aucun clone, ni bleu ni blanc. Thibault pense à un soucis avec
les cellules. Il pense aussi à la Taq ozyme, qui pourrait ne pas introduire de
\texttt{A} sortant. Mais dans ce cas, on aurait quand même des cellules bleues. Le
clonage de Raphaël ne fonctionne pas mieux. Le plus étonnant est qu'il n'y ait
pas plus de colonies bleues. Même si le clonage n'avait pas fonctionné, on
aurait quand même des cellules bleues.

\subsection{(1) Relance des clonages}
\label{sec:orgheadline71}
Les cellules qu'on avait utilisé hier sont périmées depuis septembre 2011. Ce
qui est très chiant. On refait donc le clonage avec le restant de produit de
ligature, conservé au 4°C depuis hier, en utilisant d'autres cellules, qui
périment en 2017. 

On utilise le kit de clonage TOP10. Voir description des protocoles ici :
\href{///Users/samuelbarreto/Dropbox/Cours/Master/Semestre4/StageM2/doc/invitrogen_one-shot-top10.pdf::8}{invitrogen\(_{\text{one}}\)-shot-top10.pdf, p. 8}.

Les transformants sont placés dans 300µL de milieu SOC, qui sont en étalés en 3
\texttimes{} 100µL sur les boîtes Amp75 XGal IPTG coulées hier \textit{[2016-01-20 Wed]}.

Les boîtes sont incubées 24H à 37°C. 

\subsection{(2) et (3) accusé réception séquençage}
\label{sec:orgheadline72}
\href{msgid:001501d1542e$943783b0$bca68b10$@genoscreen.fr}{GENOSCREEN : Réception des échantillons envoyés le 20/01/2016}

Genoscreen a reçu les échantillons. 

\section{2016-01-22 Vendredi}
\label{sec:orgheadline76}
\subsection{(1) Contrôle des clonages}
\label{sec:orgheadline74}
Ajouté le \textit{[2016-01-22 Fri 07:43] } \\
Beaucoup de colonies blanches et quelques colonies bleues. Le clonage d'hier
semble avoir fonctionné. On veut désormais repiquer les colonies blanches,
potentiellement transformantes, sur une autre boîte LBm Amp75 Xgal IPTG, 1) pour
confirmer leur phénotype, et 2) pour lancer des PCR de confirmation d'insertion
du fragment demain \textit{[2016-01-23 Sat]}. 

\subsection{(2) et (3) Réception des séquences}
\label{sec:orgheadline75}
Les séquences des tubes envoyés à Genoscreen ont été reçues aujourd'hui
\textit{<2016-01-22 Fri>}. 

\href{msgid:008901d15517$8e3c72b0$aab55810$@genoscreen.fr}{GENOSCREEN : Résultats des séquences du 20/01/2016 + rapport QC ( Mail 1/2)}
\href{msgid:008f01d15517$9b813500$d2839f00$@genoscreen.fr}{GENOSCREEN : Résultats des séquences du 20/01/2016 + rapport QC ( Mail 2/2)}
\section{2016-01-23 Samedi}
\label{sec:orgheadline81}
\subsection{(1) PCR contrôle des clonages}
\label{sec:orgheadline80}
Le but de cette PCR est de vérifier le sens des insertions des fragments de GS
dans pGEM-T. Avec le couple d'amorce V2 - M13R (1393 - M13R), on s'attend à
obtenir une taille d'amplification de 800bp environ, si le fragment est dans le
bon sens.

On choisit donc 6 clones confirmés \(\beta\)-gal neg par insert WS et SW. Ils sont
resuspendus dans 50µL d'eau stérile. 

\subsubsection{programme PCR}
\label{sec:orgheadline77}
Tiré du cahier de Florence, daté du \textit{[2015-06-16 Tue]}, page 135. 
\begin{center}
\begin{tabular}{lrl}
\toprule
Temps & température & 30 cycles\\
\midrule
3' & 94 & \\
50'' & 94 & x\\
50'' & 57 & x\\
50'' & 72 & x\\
4' & 72 & \\
\bottomrule
\end{tabular}
\end{center}

\subsubsection{volumes}
\label{sec:orgheadline78}
\begin{center}
\begin{tabular}{lrr}
\toprule
 & 1-tube & 14-tubes\\
\midrule
ADN & 0.5 & 7.\\
Taq & 0.5 & 7.\\
amorce M13R & 2.5 & 35.\\
amorce 1393 & 2.5 & 35.\\
MgCl2 & 1.5 & 21.\\
dNTPs & 5.0 & 70.\\
Tampon 10X Taq & 5.0 & 70.\\
\midrule
eau qsp 50µL & 32.5 & 455.\\
\bottomrule
\end{tabular}
\end{center}

\subsubsection{contrôles}
\label{sec:orgheadline79}
\begin{figure}[htb]
\centering
\includegraphics[width=\linewidth]{/Volumes/HDD/stage/img/20160123_00.JPG}
\caption{PCR 1}
\end{figure}

\begin{figure}[htb]
\centering
\includegraphics[width=\linewidth]{/Volumes/HDD/stage/img/20160123_01.JPG}
\caption{PCR 2}
\end{figure}


Rien à l'horizon. Pas même dans le témoin positif. J'ai refais la même PCR, en
passant d'un volume de 25µL à 50µL. Pas plus de chances.

Je n'ai pour l'instant pas d'explications.
\begin{enumerate}
\item les amorces sont nazes ? elles proviennent toutes les deux d'une dilution de
la solution mère, diluée ce matin pour la M13R, et mercredi \textit{[2016-01-20 Wed]}
pour la 1393.
\item la taq est naze ? J'ai essayé avec deux tubes différents.
\item le témoin positif est naze (l'insert dans ce tube n'est pas dans le bon sens)
\textbf{et} les 12 clones sont tout aussi naze ? allons allons, les lois de
l'échantillonnage ne sont quand même pas à ce point en ma défaveur…
\item ça n'est pas le bon couple d'amorce ? pourtant avec florence ça paraît pas
mal.
\item le temps d'élongation n'est pas le bon ? c'est pourtant le temps
correspondant à la taille attendue…
\end{enumerate}

Dégouté. Comment perdre son samedi en deux PCR. 


\section{2016-01-24 Dimanche}
\label{sec:orgheadline85}
\subsection{(1) Extraction Plasmidiques}
\label{sec:orgheadline82}
\subsection{(1) Ensemencements glycérols}
\label{sec:orgheadline83}


\subsection{(1) Ensemencement des tubes de LBm Amp75}
\label{sec:orgheadline84}
Deux clones par insert servent à ensemencer deux tubes de LBm Amp75, dans le but
de 1) faire les extractions plasmidiques demain \textit{<2016-01-24 Sun> } et 2) culotter
dans du glycérol, pour mettre au -80°C. 
\end{document}
               
\usepackage[utf8]{inputenc}
\usepackage[T1]{fontenc}
\usepackage{graphicx}
\usepackage{longtable}
\usepackage{float}
\usepackage{hyperref}
\usepackage{wrapfig}
\usepackage{rotating}
\usepackage[normalem]{ulem}
\usepackage{amsmath}
\usepackage{textcomp}
\usepackage{marvosym}
\usepackage{wasysym}
\usepackage{amssymb}
\usepackage[scaled=0.9]{zi4}
\usepackage[x11names, dvipsnames]{xcolor}
\usepackage[protrusion=true, expansion=alltext, tracking=true, kerning=true]{microtype}
\usepackage{siunitx}
\usepackage[french, frenchb]{babel}
\author{Samuel BARRETO}
\date{\today}
\title{}
\hypersetup{
 pdfauthor={Samuel BARRETO},
 pdftitle={},
 pdfkeywords={},
 pdfsubject={},
 pdfcreator={Emacs 24.5.1 (Org mode 8.3.3)}, 
 pdflang={Frenchb}}
\begin{document}

\setcounter{tocdepth}{2}
\tableofcontents

\clearpage

\part{2016-01 Janvier}
\label{sec:orgheadline86}
\section{2016-01-11 Lundi}
\label{sec:orgheadline6}
\subsection{(3) PCR cible des gènes synthétiques}
\label{sec:orgheadline4}
Ajouté le \textit{[2016-01-11 Mon 09:50]}

La PCR du \textit{[2016-01-07 Thu]}, cible des gènes synthétiques des clones
transformants hétérozygotes, n'a pas fonctionné, vraisemblablement dû à un
problème dans le mix PCR (pas d'amplification dans le t+). On refait donc la
même PCR sur moins de clones transformants, dans l'idée de se faire la main,
avant de faire des PCR plus conséquentes, avec de plus grands volumes. 

\subsubsection{Programme PCR :}
\label{sec:orgheadline1}
\begin{center}
\begin{tabular}{lrl}
\toprule
Temps & température & 30 cycles\\
\midrule
3' & 94 & \\
50'' & 94 & x\\
50'' & 58 & x\\
4' & 72 & x\\
\bottomrule
\end{tabular}
\end{center}

\subsubsection{Volumes :}
\label{sec:orgheadline2}
\begin{center}
\begin{tabular}{lrr}
\toprule
 & 1-tube & 7-tubes\\
\midrule
ADN & 0.5 & 3.5\\
Taq & 0.5 & 3.5\\
amorce 1392 & 2.5 & 17.5\\
amorce 1073 & 2.5 & 17.5\\
MgCl2 & 1.5 & 10.5\\
dNTPs & 5.0 & 35.\\
Tampon 10X Taq & 5.0 & 35.\\
\midrule
eau qsp 50µL & 32.5 & 227.5\\
\bottomrule
\end{tabular}
\end{center}

\subsubsection{Contrôle}
\label{sec:orgheadline3}
\begin{figure}[htb]
\centering
\includegraphics[width=0.4\linewidth]{/Volumes/HDD/stage/img/20160111_0.JPG}
\end{figure}

Ahah, la migration a duré trop longtemps à priori… Le bleu était sorti, petite
réunion avec Franck non-prévue…

On voit quand même des bandes dans les quatre puits, mais pas dans le témoin
positif, avec ADN génomique d'\emph{Acinetobacter}. Les bandes sont à la position
attendue dans le gel (taille : 750bp.) 

\begin{itemize}
\item voir à changer le témoin positif.
\end{itemize}

\subsection{(D) aliquoté Taq-ozyme}
\label{sec:orgheadline5}
Ajouté le \textit{[2016-01-11 Mon 11:47]}

Aliquoté 10µL de Taq-ozyme en eppendorf, placé dans le tiroir commun du -20°C.

\section{2016-01-12 Mardi}
\label{sec:orgheadline18}
\subsection{(3) PCR cible des gènes synthétiques}
\label{sec:orgheadline10}
Ajouté le \textit{[2016-01-12 Tue 08:19]}

La PCR d'hier ayant fonctionné, le but est de refaire la même PCR sur l'ensemble
des clones montrant des polymorphismes que j'ai isolé, avant de pouvoir les
envoyer à séquencer.

\begin{itemize}
\item Penser à changer de témoin positif (peut-être utiliser un plasmide porteur de
l'une des constructions par exemple.)
\end{itemize}

\subsubsection{Programme PCR :}
\label{sec:orgheadline7}
\begin{center}
\begin{tabular}{lrl}
\toprule
Temps & température & 30 cycles\\
\midrule
3' & 94 & \\
50'' & 94 & x\\
50'' & 58 & x\\
50'' & 72 & x\\
4' & 72 & \\
\bottomrule
\end{tabular}
\end{center}

\subsubsection{Volumes :}
\label{sec:orgheadline8}
\begin{center}
\begin{tabular}{lrr}
\toprule
 & 1-tube & 27-tubes\\
\midrule
ADN & 0.5 & 13.5\\
Taq & 0.5 & 13.5\\
amorce 1392 & 2.5 & 67.5\\
amorce 1073 & 2.5 & 67.5\\
MgCl2 & 1.5 & 40.5\\
dNTPs & 5.0 & 135.\\
Tampon 10X Taq & 5.0 & 135.\\
\midrule
eau qsp 50µL & 32.5 & 877.5\\
\bottomrule
\end{tabular}
\end{center}

\subsubsection{Contrôle :}
\label{sec:orgheadline9}

\begin{figure}[htb]
\centering
\includegraphics[width=3cm]{/Volumes/HDD/stage/img/20160112_00.JPG}
\caption{Ladder un peu chargé. Attention au volume. Colonies 3 et 4 de pW5 et 3 de pS53 n'ont pas amplifié. Témoin positif migration sur le gel de \textit{[2016-01-12 Tue 16:05]}.}
\end{figure}

\begin{itemize}
\item $\square$ refaire la PCR pour les deux clones pW5 et le clone de pS53 qui n'a pas
fonctionné.
\item $\square$ faire migrer le témoin positif.
\end{itemize}

\subsection{(2) PCR cible de l'ancre}
\label{sec:orgheadline15}
Ajouté le \textit{[2016-01-12 Tue 14:15]}

Chez les différents clones montrant des néomutations, on veut séquencer l'ancre,
pour vérifier que le taux de mutation dans la tract de conversion est bien
supérieur au taux de mutation dans une zone parfaitement homologue. L'hypothèse,
floue pour l'instant, est que le système de réparation des mésappariemments est
"saturé", et qu'il introduit des erreurs. (Ces mutations semblent biaisées vers
GC d'après les données dont on dispose pour l'instant.)

Le but est ici de réaliser une PCR simple sur quelques candidats, avant de
passer aux grands volumes, envoyés à séquencer. Les conditions de PCR sont
tirées du cahier de Florence, page 146, daté du \textit{[2015-06-25 Thu]}. 

\subsubsection{Candidats sélectionnés}
\label{sec:orgheadline11}
\begin{center}
\begin{tabular}{ll}
\toprule
strong & weak\\
\midrule
pS10 & pW14\\
pS24 & pW19\\
\bottomrule
\end{tabular}
\end{center}

\subsubsection{Programme PCR}
\label{sec:orgheadline12}
\begin{center}
\begin{tabular}{lrl}
\toprule
Temps & température & 30 cycles\\
\midrule
3' & 94 & \\
50'' & 94 & x\\
50'' & 55 & x\\
60'' & 72 & x\\
4' & 72 & \\
\bottomrule
\end{tabular}
\end{center}

\subsubsection{Volumes :}
\label{sec:orgheadline13}
\begin{center}
\begin{tabular}{lrr}
\toprule
 & 1-tube & 7-tubes\\
\midrule
ADN & 0.5 & 3.5\\
Taq & 0.5 & 3.5\\
amorce 1410 & 2.5 & 17.5\\
amorce 1411 & 2.5 & 17.5\\
MgCl2 & 1.5 & 10.5\\
dNTPs & 5.0 & 35.\\
Tampon 10X Taq & 5.0 & 35.\\
\midrule
eau qsp 50µL & 32.5 & 227.5\\
\bottomrule
\end{tabular}
\end{center}

\subsubsection{Contrôle :}
\label{sec:orgheadline14}

\begin{figure}[htb]
\centering
\includegraphics[width=4cm]{/Volumes/HDD/stage/img/20160113_00.JPG}
\caption{\label{fig:orgparagraph1}
Le contrôle de la PCR cible de l'ancre 100\% homologue. Le témoin positif de la PCR d'hier est censé être là également. }
\end{figure}

Voir figure \ref{fig:orgparagraph1}. 

\begin{itemize}
\item $\square$ Refaire migrer le témoin positif.
\end{itemize}

\subsection{(D) dilué amorces}
\label{sec:orgheadline16}
Ajouté le \textit{[2016-01-12 Tue 10:02]}

Dilué au 10\(^{\text{e}}\) (10 / 90µL eau aguettant) les amorces 1073, 1392, 1410 et 1411.

\subsection{(D) coulé gel}
\label{sec:orgheadline17}
Ajouté le \textit{[2016-01-12 Tue 16:01]}

Coulé deux grands gels 300mL agarose 1\% 26 puits.
\section{2016-01-13 Mercredi}
\label{sec:orgheadline29}
\subsection{(2) PCR cible des ancres}
\label{sec:orgheadline23}
Ajouté le \textit{[2016-01-13 Wed 09:36]}

La PCR d'hier ayant fonctionné, le but est de faire la même PCR dans les mêmes
conditions pour séquencer toutes les ancres des clones montrant des
néomutations. 

\subsubsection{candidats sélectionnés}
\label{sec:orgheadline19}
\begin{center}
\begin{tabular}{ll}
\toprule
strong & weak\\
\midrule
pS10 & pW14\\
pS24 & pW19\\
pS30 & pW2\\
pS39 & pW23\\
pS5 & pW35\\
pS54 & pW6\\
pS74 & pW81\\
pS82 & pW87\\
pS88 & pW93\\
\bottomrule
\end{tabular}
\end{center}

\begin{itemize}
\item erreur : trompé de clone, prélevé pW5 au lieu de pW2… gros malin.
\end{itemize}

\subsubsection{programme PCR}
\label{sec:orgheadline20}
\begin{center}
\begin{tabular}{lrl}
\toprule
Temps & température & 30 cycles\\
\midrule
3' & 94 & \\
50'' & 94 & x\\
50'' & 55 & x\\
60'' & 72 & x\\
4' & 72 & \\
\bottomrule
\end{tabular}
\end{center}

\subsubsection{volumes}
\label{sec:orgheadline21}
\begin{center}
\begin{tabular}{lrr}
\toprule
 & 1-tube & 19-tubes\\
\midrule
ADN & 0.5 & 9.5\\
Taq & 0.5 & 9.5\\
amorce 1410 & 2.5 & 47.5\\
amorce 1411 & 2.5 & 47.5\\
MgCl2 & 1.5 & 28.5\\
dNTPs & 5.0 & 95.\\
Tampon 10X Taq & 5.0 & 95.\\
\midrule
eau qsp 50µL & 32.5 & 617.5\\
\bottomrule
\end{tabular}
\end{center}

\subsubsection{contrôle}
\label{sec:orgheadline22}
Voir figure \ref{fig:orgparagraph2}. 

\begin{figure}[htb]
\centering
\includegraphics[width=\linewidth]{/Volumes/HDD/stage/img/20160113_01.JPG}
\caption{\label{fig:orgparagraph2}
La PCR cible des ancres semble avoir fonctionné de partout. Dans l'ordre, de pS10 à pS88, puis pW14 à pW93, T- et T+. Le T- est clean, le T+ trop fort. Penser à moins déposer pour le T+.}
\end{figure}

Tout a fonctionné a priori. Les produits de PCR sont conservés au -20°C dans le
portoir orange à couvercle. Je refais la PCR pour l'isolat pW2.

\subsection{(3) PCR cible des gènes synthétiques}
\label{sec:orgheadline28}
Il n'y a pas de bandes dans trois cas d'amplification, l'hypothèse la plus
parcimonieuse étant que la PCR n'a pas fonctionné… Je la refais donc pour ces
trois isolats là. lala. 

\subsubsection{candidats}
\label{sec:orgheadline24}
pW5\(_{\text{3}}\) -- pW5\(_{\text{4}}\) -- pS53\(_{\text{3}}\) 

\subsubsection{programme PCR}
\label{sec:orgheadline25}
\begin{center}
\begin{tabular}{lrl}
\toprule
Temps & température & 30 cycles\\
\midrule
3' & 94 & \\
50'' & 94 & x\\
50'' & 58 & x\\
50'' & 72 & x\\
4' & 72 & \\
\bottomrule
\end{tabular}
\end{center}

\subsubsection{volumes}
\label{sec:orgheadline26}
\begin{center}
\begin{tabular}{lrr}
\toprule
 & 1-tube & 6-tubes\\
\midrule
ADN & 0.5 & 3.\\
Taq & 0.5 & 3.\\
amorce 1410 & 2.5 & 15.\\
amorce 1411 & 2.5 & 15.\\
MgCl2 & 1.5 & 9.\\
dNTPs & 5.0 & 30.\\
Tampon 10X Taq & 5.0 & 30.\\
\midrule
eau qsp 50µL & 32.5 & 195.\\
\bottomrule
\end{tabular}
\end{center}

\subsubsection{contrôle}
\label{sec:orgheadline27}
Voir figure \ref{fig:orgparagraph3}. 
Absence d'amplification. Repartir des boîtes, resuspendre les colonies, et
refaire la PCR. 

\begin{figure}[htb]
\centering
\includegraphics[width=0.3\linewidth]{/Volumes/HDD/stage/img/20160113_02.JPG}
\caption{\label{fig:orgparagraph3}
Rien à l'horizon. Ladder tordu, puisque gel qui se tortille dans la cuve… Bien fait pour moi.}
\end{figure}

\section{2016-01-14 Jeudi}
\label{sec:orgheadline38}
\subsection{(3) PCR des trois clones récalcitrants}
\label{sec:orgheadline34}
La PCR d'hier semble montrer qu'il n'y a pas d'ADN dans le tube. Le témoin
positif, qui fonctionne bien avec le couple d'amorce 1410--1411 ne fonctionne
pas avec les amorces 1073--1392, pour une raison que je ne m'explique encore
pas. 

Je veux donc repartir d'une nouvelle suspension, et refaire (encore une fois) la
PCR sur ces trois colonies. Le but est d'avoir un plan d'expérience équilibré,
avec quatre colonies isolées par candidat choisi.
\subsubsection{suspension des clones}
\label{sec:orgheadline30}
Les trois colonies sélectionnées, pW5\(_{\text{3}}\) -- pW5\(_{\text{4}}\) -- pS53\(_{\text{3}}\), sont resuspendues
dans 20µL d'eau pure. 
\subsubsection{programme PCR}
\label{sec:orgheadline31}
\begin{center}
\begin{tabular}{lrl}
\toprule
Temps & température & 30 cycles\\
\midrule
3' & 94 & \\
50'' & 94 & x\\
50'' & 58 & x\\
50'' & 72 & x\\
4' & 72 & \\
\bottomrule
\end{tabular}
\end{center}

\subsubsection{volumes}
\label{sec:orgheadline32}
\begin{center}
\begin{tabular}{lrr}
\toprule
 & 1-tube & 6-tubes\\
\midrule
ADN & 0.5 & 3.\\
Taq & 0.5 & 3.\\
amorce 1410 & 2.5 & 15.\\
amorce 1411 & 2.5 & 15.\\
MgCl2 & 1.5 & 9.\\
dNTPs & 5.0 & 30.\\
Tampon 10X Taq & 5.0 & 30.\\
\midrule
eau qsp 50µL & 32.5 & 195.\\
\bottomrule
\end{tabular}
\end{center}

\subsubsection{contrôle}
\label{sec:orgheadline33}

\begin{figure}[htb]
\centering
\includegraphics[width=0.5\linewidth]{/Volumes/HDD/stage/img/20160114_00.JPG}
\caption{Les trois clones ont bien amplifié. Le T+ a toujours le même soucis, je suis pourtant parti d'un nouvel ADN génomique d'Acineto.}
\end{figure}

\subsection{(1) Contrôle des extractions plasmidiques}
\label{sec:orgheadline36}
Les extractions des plasmides porteurs des gènes synthétiques du
\textit{[2016-01-08 Fri] } sont contrôlées au nano-drop.

Résultat :
\begin{center}
\begin{tabular}{ll}
\toprule
plasmide & dosage\\
\midrule
WS & 8.5ng/µL\\
SW & 7.2ng/µL\\
\bottomrule
\end{tabular}
\end{center}

Il n'y a pratiquement rien. 

\subsubsection{relance des cultures}
\label{sec:orgheadline35}
J'ai sorti les cryotubes du -80°C, resuspendu les cultures dans 5mL de LB+Kan50,
incubation 24H à 37°C. Le but est de refaire les extractions demain
\textit{[2016-01-15 Fri] } avec plus de milieu, en culottant les 5mL de culture, et en
éluant dans un plus petit volume.

\subsection{(D) maintenance}
\label{sec:orgheadline37}
\begin{itemize}
\item nettoyé paillasse
\item aliquoté 10 \texttimes{} 1.5mL d'eau UP en eppendorf.
\end{itemize}
\section{2016-01-15 Vendredi}
\label{sec:orgheadline44}
\subsection{(1) Extractions plasmidiques}
\label{sec:orgheadline42}
\subsubsection{Extractions}
\label{sec:orgheadline39}
Vu les dosages nanodrop des extractions précédentes, je refais l'extraction sur
une culture liquide fraîche. Culotté 5mL et suivi le protocole standard. Élution
en deux étapes (2\texttimes{}15µL), incubation 3min à °C ambiante. Les tubes sont
appelés \texttt{WS} et \texttt{SW}, \textit{[2016-01-15 Fri]}.

\subsubsection{Contrôle}
\label{sec:orgheadline40}

\begin{figure}[htb]
\centering
\includegraphics[width=0.25\linewidth]{/Volumes/HDD/stage/img/20160115_00.JPG}
\caption{Le plasmide WS semble bon, il ne smirre pas trop. Le plasmide SW est à refaire. }
\end{figure}

\subsubsection{NanoDrop}
\label{sec:orgheadline41}
\begin{center}
\begin{tabular}{lr}
\toprule
 & dosage (ng/µL)\\
\midrule
WS & 25.2\\
 & 25.8\\
\midrule
SW & 6.7\\
 & 5.9\\
 & 9.1\\
 & 5.0\\
 & nul quoi. 0\\
\bottomrule
\end{tabular}
\end{center}
\subsection{(A) Analyses}
\label{sec:orgheadline43}
Avancé sur les alignements polySNP, sur l'extraction des données comme vu avec
Laurent hier \textit{[2016-01-14 Thu]}. 
\section{2016-01-18 Lundi}
\label{sec:orgheadline52}
\subsection{(1) Constructions In-Fusion}
\label{sec:orgheadline45}
On veut mettre au point une méthode de clonage par PCR qui permettrait d'accoler
les fragments GS (gène synthétique) avec apha3 (kanamycine) et l'ancre.

Les kits In-Fusion permettent de concevoir ça in-silico.

Le vecteur utilisé est le plasmide pGEMT-T. Il est ouvert par digestion SpeI. Il
faut absence de site de restriction SpeI dans le GS. Ç'est le cas. Vérifié le
\textit{<2016-01-18 Mon 08:54>}.

\begin{figure}[htb]
\centering
\includegraphics[width=\linewidth]{/Volumes/HDD/stage/img/20160118_00.png}
\caption{Fragment 1 : le Gene Synthétique. Fragment 2 : le gène aphA-3. Fragment 2 : le gène ancre.}
\end{figure}
\subsection{(2) PCR pW2 avant sequençage}
\label{sec:orgheadline48}
L'isolât pW2 n'a pas été amplifié, confondu les tubes (voir figure
\ref{fig:orgparagraph2}). On refait la PCR cible ancre sur cet isolat là. 

\subsubsection{programme PCR}
\label{sec:orgheadline46}
\begin{center}
\begin{tabular}{lrl}
\toprule
Temps & température & 30 cycles\\
\midrule
3' & 94 & \\
50'' & 94 & x\\
50'' & 55 & x\\
60'' & 72 & x\\
4' & 72 & \\
\bottomrule
\end{tabular}
\end{center}

\subsubsection{volumes}
\label{sec:orgheadline47}
\begin{center}
\begin{tabular}{lrr}
\toprule
 & 1-tube & 4-tubes\\
\midrule
ADN & 0.5 & 2.\\
Taq & 0.5 & 2.\\
amorce 1410 & 2.5 & 10.\\
amorce 1411 & 2.5 & 10.\\
MgCl2 & 1.5 & 6.\\
dNTPs & 5.0 & 20.\\
Tampon 10X Taq & 5.0 & 20.\\
\midrule
eau qsp 50µL & 32.5 & 130.\\
\bottomrule
\end{tabular}
\end{center}

\subsection{(2) Envoi ancres sequençage.}
\label{sec:orgheadline49}
Les produits PCR du \textit{[2016-01-13 Wed] } sont envoyés à séquencer. But : séquencer
les ancres des clones qui montrent des néomutations. 
\subsection{(3) Envoi produits PCR sequencage.}
\label{sec:orgheadline50}
Les produits PCR du \textit{[2016-01-12 Tue] } sont envoyés à séquencer. But : séquencer
les gènes synthétiques de certains candidats qui montrent des polymorphismes
assez marqués. 
\subsection{(D) Aliquoté amorces}
\label{sec:orgheadline51}
Aliquoté 3\texttimes{} 100mL des amorces amorces \texttt{1073 1392 1411 1410} en diluant au
10\(^{\text{e}}\).
\section{2016-01-19 Mardi}
\label{sec:orgheadline58}
\subsection{(1) PCR cible GS}
\label{sec:orgheadline56}
But : insérer les gènes synthétiques alternants dans pGEM-T, après une PCR \texttt{A}
sortant. 
\subsubsection{Programme PCR :}
\label{sec:orgheadline53}
\begin{center}
\begin{tabular}{lrl}
\toprule
Temps & température & 30 cycles\\
\midrule
3' & 94 & \\
50'' & 94 & x\\
50'' & 58 & x\\
50'' & 72 & x\\
4' & 72 & \\
\bottomrule
\end{tabular}
\end{center}
\subsubsection{Volumes :}
\label{sec:orgheadline54}
\begin{center}
\begin{tabular}{lrr}
\toprule
 & 1-tube & 7-tubes\\
\midrule
ADN & 0.5 & 3.5\\
Taq & 0.5 & 3.5\\
amorce 1392 & 2.5 & 17.5\\
amorce 1073 & 2.5 & 17.5\\
MgCl2 & 1.5 & 10.5\\
dNTPs & 5.0 & 35.\\
Tampon 10X Taq & 5.0 & 35.\\
\midrule
eau qsp 50µL & 32.5 & 227.5\\
\bottomrule
\end{tabular}
\end{center}

Problème : pas le bon couple d'amorce. 1073 est une amorce spécifique \emph{Acineto}.
Le couple d'amorce nécessaire est le couple \texttt{1392 1393}. La PCR est refaite
demain \textit{<2016-01-20 Wed 07:00>}.

\subsubsection{contrôle}
\label{sec:orgheadline55}
Comme attendu, pas de bandes. En plus la PCR est un peu sale. Penser à changer
l'eau. 


\begin{figure}[htb]
\centering
\includegraphics[width=3cm]{/Volumes/HDD/stage/img/20160119_00.JPG}
\caption{ }
\end{figure}


\subsection{(1) Couler boîtes Amp75 XGal IPTG}
\label{sec:orgheadline57}
Ces boîtes servent à sélectionner les transformants qui possèdent le gène
synthétique dans le site de clonage. La disruption du gène de la \(\beta\)-gal par
l'insertion donne des colonies blanches résistantes à l'ampicilline.

Couler 10 boîtes de LBm aux concentrations :
\begin{center}
\begin{tabular}{lll}
\toprule
 & [] & V\(_{\text{0}}\)\\
\midrule
Xgal & 60µg/mL & 115 µL\\
IPTG & 40µg/mL & 0.5 mL\\
Amp75 & 75µg/mL & 0.25mL\\
\bottomrule
\end{tabular}
\end{center}

\section{2016-01-20 Mercredi}
\label{sec:orgheadline69}
\subsection{(1) PCR cible GS}
\label{sec:orgheadline62}
Le but est d'amplifier les fragments synthétiques, dans l'idée de les insérer
après purification dans le plasmide pGEM-T. La PCR donne des \texttt{A} sortants,
ligaturés dans les \texttt{T} sortant du plasmide pGEM-T. 

\subsubsection{Programme PCR :}
\label{sec:orgheadline59}
D'après le cahier de Florence, daté du \textit{[2015-05-15 Fri]}, page 120. 
\begin{center}
\begin{tabular}{lrl}
\toprule
Temps & température & 30 cycles\\
\midrule
3' & 94 & \\
50'' & 94 & x\\
50'' & 60 & x\\
45'' & 72 & x\\
4' & 72 & \\
\bottomrule
\end{tabular}
\end{center}
\subsubsection{Volumes :}
\label{sec:orgheadline60}
\begin{center}
\begin{tabular}{lrr}
\toprule
 & 1-tube & 7-tubes\\
\midrule
ADN & 0.5 & 3.5\\
Taq & 0.5 & 3.5\\
amorce 1392 & 2.5 & 17.5\\
amorce 1393 & 2.5 & 17.5\\
MgCl2 & 1.5 & 10.5\\
dNTPs & 5.0 & 35.\\
Tampon 10X Taq & 5.0 & 35.\\
\midrule
eau qsp 50µL & 32.5 & 227.5\\
\bottomrule
\end{tabular}
\end{center}

Le T+ utilisé est un plasmide pGEM-T porteur de la construction Weak, synthétisé
par Florence. 
\subsubsection{contrôle}
\label{sec:orgheadline61}
\begin{figure}[htb]
\centering
\includegraphics[width=0.5\linewidth]{/Volumes/HDD/stage/img/20160120_00.JPG}
\caption{Contrôle de la PCR 1392-1393. Le témoin positif est un pGEM-T de Florence porteur de la construction Weak. }
\end{figure}

Tout est good. On peut lancer la purification, et éluer dans un bon volume de
25µL. 
\subsection{(1) Purification des fragments PCR amplifiés}
\label{sec:orgheadline64}
Les fragments PCR sont purifiés par le protocole
\href{///Users/samuelbarreto/Dropbox/Cours/Master/Semestre4/StageM2/doc/nucleospin_pcr_purif_cleanup.pdf::17}{nucleospin\(_{\text{pcr}}_{\text{purif}}_{\text{cleanup.pdf}}\), p. 17}. 

Les deux produits de PCR SW et WS sont poolés avant d'être mélangé au tampon de
charge de l'étape 1. L'éthanol à l'étape 4 est laissé à évaporer pendant 5min à
60°C, colonne ouverte dans le bloc chauffant. L'ADN est élué par un volume de
25µL par de l'eau UP chauffée préalablement à 60°C.

\subsubsection{Contrôle}
\label{sec:orgheadline63}

\begin{figure}[htb]
\centering
\includegraphics[width=0.3\linewidth]{/Volumes/HDD/stage/img/20160120_01.JPG}
\caption{Contrôle de la purification des produits PCR. Les deux purifications semblent avoir fonctionné comme il faut. 2µL sont chargés. 3µL de ladder ne suffisent pas. Il faut homogénéiser le ladder et charger plus.}
\end{figure}
\subsection{(1) Ligature dans pGEM-T}
\label{sec:orgheadline65}
Le plasmide pGEM-T de Promega est un plasmide conçu pour avoir des extrémités
franches porteuses de bases \texttt{T}. Il est maintenu ouvert dans son tampon,
conservé à -20°C. 

Le but est de ligaturer les fragments PCR obtenus et purifiés avec le pGEM-T,
puis de transformer des cellules compétentes \emph{E.coli} par choc chaud-froid. 

Le protocole utilisé est le protocole promega pGEM-T (pas pGEM-T easy). Lien
pour le short-manual : \href{///Users/samuelbarreto/Dropbox/Cours/Master/Semestre4/StageM2/doc/promega_pgem-t_short-manual.pdf::1}{promega\(_{\text{pgem}}\)-t\(_{\text{short}}\)-manual.pdf, p. 1}. Lien pour le
protocole complet : \href{///Users/samuelbarreto/Dropbox/Cours/Master/Semestre4/StageM2/doc/promega_pgem-t.pdf::5}{promega\(_{\text{pgem}}\)-t.pdf, p. 5}.

\subsection{(1) Transformation dans \emph{E.coli} TOP10}
\label{sec:orgheadline66}
Les produits de ligature sont utilisés pour transformer des \emph{E.coli} TOP10
thermocompétentes à la transformation. Le reste des produits de ligature est
placé au 4°C sur la nuit, pour favoriser l'apparition de transformants, en cas
de soucis.

Les transformants sont placés dans 300µL de milieu SOC, qui sont en étalés en 3
\texttimes{} 100µL sur les boîtes Amp75 XGal IPTG coulées hier \textit{[2016-01-19 Tue]}.

Les boîtes sont incubées 24H à 37°C. 

\subsection{(2) et (3) envoi au séquençage}
\label{sec:orgheadline67}
Les tubes à envoyer à séquencer, contenant les produits PCR cibles des gènes
synthétiques des clones présentant des polymorphismes et des ancres des clones
présentant des néomutations ont été envoyés à séquencer par David aujourd'hui
\textit{[2016-01-21 Thu]}. 

\subsection{(D) coulé boîtes amp-xgal-iptg}
\label{sec:orgheadline68}
Coulé 10 boîtes ampicilline 75, Xgal IPTG, placées en chambre froide.
\section{2016-01-21 Jeudi}
\label{sec:orgheadline73}
\subsection{(1) Contrôle des clonages}
\label{sec:orgheadline70}
Pas de culture. Aucun clone, ni bleu ni blanc. Thibault pense à un soucis avec
les cellules. Il pense aussi à la Taq ozyme, qui pourrait ne pas introduire de
\texttt{A} sortant. Mais dans ce cas, on aurait quand même des cellules bleues. Le
clonage de Raphaël ne fonctionne pas mieux. Le plus étonnant est qu'il n'y ait
pas plus de colonies bleues. Même si le clonage n'avait pas fonctionné, on
aurait quand même des cellules bleues.

\subsection{(1) Relance des clonages}
\label{sec:orgheadline71}
Les cellules qu'on avait utilisé hier sont périmées depuis septembre 2011. Ce
qui est très chiant. On refait donc le clonage avec le restant de produit de
ligature, conservé au 4°C depuis hier, en utilisant d'autres cellules, qui
périment en 2017. 

On utilise le kit de clonage TOP10. Voir description des protocoles ici :
\href{///Users/samuelbarreto/Dropbox/Cours/Master/Semestre4/StageM2/doc/invitrogen_one-shot-top10.pdf::8}{invitrogen\(_{\text{one}}\)-shot-top10.pdf, p. 8}.

Les transformants sont placés dans 300µL de milieu SOC, qui sont en étalés en 3
\texttimes{} 100µL sur les boîtes Amp75 XGal IPTG coulées hier \textit{[2016-01-20 Wed]}.

Les boîtes sont incubées 24H à 37°C. 

\subsection{(2) et (3) accusé réception séquençage}
\label{sec:orgheadline72}
\href{msgid:001501d1542e$943783b0$bca68b10$@genoscreen.fr}{GENOSCREEN : Réception des échantillons envoyés le 20/01/2016}

Genoscreen a reçu les échantillons. 

\section{2016-01-22 Vendredi}
\label{sec:orgheadline76}
\subsection{(1) Contrôle des clonages}
\label{sec:orgheadline74}
Ajouté le \textit{[2016-01-22 Fri 07:43] } \\
Beaucoup de colonies blanches et quelques colonies bleues. Le clonage d'hier
semble avoir fonctionné. On veut désormais repiquer les colonies blanches,
potentiellement transformantes, sur une autre boîte LBm Amp75 Xgal IPTG, 1) pour
confirmer leur phénotype, et 2) pour lancer des PCR de confirmation d'insertion
du fragment demain \textit{[2016-01-23 Sat]}. 

\subsection{(2) et (3) Réception des séquences}
\label{sec:orgheadline75}
Les séquences des tubes envoyés à Genoscreen ont été reçues aujourd'hui
\textit{<2016-01-22 Fri>}. 

\href{msgid:008901d15517$8e3c72b0$aab55810$@genoscreen.fr}{GENOSCREEN : Résultats des séquences du 20/01/2016 + rapport QC ( Mail 1/2)}
\href{msgid:008f01d15517$9b813500$d2839f00$@genoscreen.fr}{GENOSCREEN : Résultats des séquences du 20/01/2016 + rapport QC ( Mail 2/2)}
\section{2016-01-23 Samedi}
\label{sec:orgheadline81}
\subsection{(1) PCR contrôle des clonages}
\label{sec:orgheadline80}
Le but de cette PCR est de vérifier le sens des insertions des fragments de GS
dans pGEM-T. Avec le couple d'amorce V2 - M13R (1393 - M13R), on s'attend à
obtenir une taille d'amplification de 800bp environ, si le fragment est dans le
bon sens.

On choisit donc 6 clones confirmés \(\beta\)-gal neg par insert WS et SW. Ils sont
resuspendus dans 50µL d'eau stérile. 

\subsubsection{programme PCR}
\label{sec:orgheadline77}
Tiré du cahier de Florence, daté du \textit{[2015-06-16 Tue]}, page 135. 
\begin{center}
\begin{tabular}{lrl}
\toprule
Temps & température & 30 cycles\\
\midrule
3' & 94 & \\
50'' & 94 & x\\
50'' & 57 & x\\
50'' & 72 & x\\
4' & 72 & \\
\bottomrule
\end{tabular}
\end{center}

\subsubsection{volumes}
\label{sec:orgheadline78}
\begin{center}
\begin{tabular}{lrr}
\toprule
 & 1-tube & 14-tubes\\
\midrule
ADN & 0.5 & 7.\\
Taq & 0.5 & 7.\\
amorce M13R & 2.5 & 35.\\
amorce 1393 & 2.5 & 35.\\
MgCl2 & 1.5 & 21.\\
dNTPs & 5.0 & 70.\\
Tampon 10X Taq & 5.0 & 70.\\
\midrule
eau qsp 50µL & 32.5 & 455.\\
\bottomrule
\end{tabular}
\end{center}

\subsubsection{contrôles}
\label{sec:orgheadline79}
\begin{figure}[htb]
\centering
\includegraphics[width=\linewidth]{/Volumes/HDD/stage/img/20160123_00.JPG}
\caption{PCR 1}
\end{figure}

\begin{figure}[htb]
\centering
\includegraphics[width=\linewidth]{/Volumes/HDD/stage/img/20160123_01.JPG}
\caption{PCR 2}
\end{figure}


Rien à l'horizon. Pas même dans le témoin positif. J'ai refais la même PCR, en
passant d'un volume de 25µL à 50µL. Pas plus de chances.

Je n'ai pour l'instant pas d'explications.
\begin{enumerate}
\item les amorces sont nazes ? elles proviennent toutes les deux d'une dilution de
la solution mère, diluée ce matin pour la M13R, et mercredi \textit{[2016-01-20 Wed]}
pour la 1393.
\item la taq est naze ? J'ai essayé avec deux tubes différents.
\item le témoin positif est naze (l'insert dans ce tube n'est pas dans le bon sens)
\textbf{et} les 12 clones sont tout aussi naze ? allons allons, les lois de
l'échantillonnage ne sont quand même pas à ce point en ma défaveur…
\item ça n'est pas le bon couple d'amorce ? pourtant avec florence ça paraît pas
mal.
\item le temps d'élongation n'est pas le bon ? c'est pourtant le temps
correspondant à la taille attendue…
\end{enumerate}

Dégouté. Comment perdre son samedi en deux PCR. 


\section{2016-01-24 Dimanche}
\label{sec:orgheadline85}
\subsection{(1) Extraction Plasmidiques}
\label{sec:orgheadline82}
\subsection{(1) Ensemencements glycérols}
\label{sec:orgheadline83}


\subsection{(1) Ensemencement des tubes de LBm Amp75}
\label{sec:orgheadline84}
Deux clones par insert servent à ensemencer deux tubes de LBm Amp75, dans le but
de 1) faire les extractions plasmidiques demain \textit{<2016-01-24 Sun> } et 2) culotter
dans du glycérol, pour mettre au -80°C. 
\end{document}
               
\usepackage[utf8]{inputenc}
\usepackage[T1]{fontenc}
\usepackage{graphicx}
\usepackage{longtable}
\usepackage{float}
\usepackage{hyperref}
\usepackage{wrapfig}
\usepackage{rotating}
\usepackage[normalem]{ulem}
\usepackage{amsmath}
\usepackage{textcomp}
\usepackage{marvosym}
\usepackage{wasysym}
\usepackage{amssymb}
\usepackage[scaled=0.9]{zi4}
\usepackage[x11names, dvipsnames]{xcolor}
\usepackage[protrusion=true, expansion=alltext, tracking=true, kerning=true]{microtype}
\usepackage{siunitx}
\usepackage[frenchle, frenchb]{babel}
\author{Samuel BARRETO}
\date{\today}
\title{}
\hypersetup{
 pdfauthor={Samuel BARRETO},
 pdftitle={},
 pdfkeywords={},
 pdfsubject={},
 pdfcreator={Emacs 24.5.1 (Org mode 8.3.2)}, 
 pdflang={Frenchb}}
\begin{document}


\part{2016-01 Janvier}
\label{sec:orgheadline39}
\section{2016-01-11 Lundi}
\label{sec:orgheadline6}
\subsection{(3) PCR cible des gènes synthétiques}
\label{sec:orgheadline4}
Ajouté le \textit{[2016-01-11 Mon 09:50]}

La PCR du \textit{[2016-01-07 Thu]}, cible des gènes synthétiques des clones
transformants hétérozygotes, n'a pas fonctionné, vraisemblablement dû à un
problème dans le mix PCR (pas d'amplification dans le t+). On refait donc la
même PCR sur moins de clones transformants, dans l'idée de se faire la main,
avant de faire des PCR plus conséquentes, avec de plus grands volumes. 

\subsubsection{Programme PCR :}
\label{sec:orgheadline1}
\begin{center}
\begin{tabular}{lrl}
\toprule
Temps & température & 30 cycles\\
\midrule
3' & 94 & \\
50'' & 94 & x\\
50'' & 58 & x\\
4' & 72 & x\\
\bottomrule
\end{tabular}
\end{center}

\subsubsection{Volumes :}
\label{sec:orgheadline2}
\begin{center}
\begin{tabular}{lrr}
\toprule
 & 1-tube & 7-tubes\\
\midrule
ADN & 0.5 & 3.5\\
Taq & 0.5 & 3.5\\
amorce 1392 & 2.5 & 17.5\\
amorce 1073 & 2.5 & 17.5\\
MgCl2 & 1.5 & 10.5\\
dNTPs & 5.0 & 35.\\
Tampon 10X Taq & 5.0 & 35.\\
\midrule
eau qsp 50µL & 32.5 & 227.5\\
\bottomrule
\end{tabular}
\end{center}

\subsubsection{Contrôle}
\label{sec:orgheadline3}
\includegraphics[width=0.7\linewidth]{/Volumes/HDD/stage/img/20160111_0.JPG}

\includegraphics[width=0.4\linewidth]{/Users/samuelbarreto/Dropbox/Cours/Master/Semestre4/StageM2/gene-ruler-1kb.jpg}

Ahah, la migration a duré trop longtemps à priori… Le bleu était sorti, petite
réunion avec Franck non-prévue…

On voit quand même des bandes dans les quatre puits, mais pas dans le témoin
positif, avec ADN génomique d'\emph{Acinetobacter}. Les bandes sont à la position
attendue dans le gel (taille : 750bp.) 

\begin{itemize}
\item voir à changer le témoin positif.
\end{itemize}

\subsection{(D) aliquoté Taq-ozyme}
\label{sec:orgheadline5}
Ajouté le \textit{[2016-01-11 Mon 11:47]}

Aliquoté 10µL de Taq-ozyme en eppendorf, placé dans le tiroir commun du -20°C.

\section{2016-01-12 Mardi}
\label{sec:orgheadline18}
\subsection{(3) PCR cible des gènes synthétiques}
\label{sec:orgheadline10}
Ajouté le \textit{[2016-01-12 Tue 08:19]}

La PCR d'hier ayant fonctionné, le but est de refaire la même PCR sur l'ensemble
des clones montrant des polymorphismes que j'ai isolé, avant de pouvoir les
envoyer à séquencer.

\begin{itemize}
\item Penser à changer de témoin positif (peut-être utiliser un plasmide porteur de
l'une des constructions par exemple.)
\end{itemize}

\subsubsection{Programme PCR :}
\label{sec:orgheadline7}
\begin{center}
\begin{tabular}{lrl}
\toprule
Temps & température & 30 cycles\\
\midrule
3' & 94 & \\
50'' & 94 & x\\
50'' & 58 & x\\
50'' & 72 & x\\
4' & 72 & \\
\bottomrule
\end{tabular}
\end{center}

\subsubsection{Volumes :}
\label{sec:orgheadline8}
\begin{center}
\begin{tabular}{lrr}
\toprule
 & 1-tube & 27-tubes\\
\midrule
ADN & 0.5 & 13.5\\
Taq & 0.5 & 13.5\\
amorce 1392 & 2.5 & 67.5\\
amorce 1073 & 2.5 & 67.5\\
MgCl2 & 1.5 & 40.5\\
dNTPs & 5.0 & 135.\\
Tampon 10X Taq & 5.0 & 135.\\
\midrule
eau qsp 50µL & 32.5 & 877.5\\
\bottomrule
\end{tabular}
\end{center}

\subsubsection{Contrôle :}
\label{sec:orgheadline9}

\begin{figure}[htb]
\centering
\includegraphics[width=3cm]{/Volumes/HDD/stage/img/20160112_00.JPG}
\caption{Ladder un peu chargé. Attention au volume. Colonies 3 et 4 de pW5 et 3 de pS53 n'ont pas amplifié. Témoin positif migration sur le gel de \textit{[2016-01-12 Tue 16:05]}.}
\end{figure}

\begin{itemize}
\item $\square$ refaire la PCR pour les deux clones pW5 et le clone de pS53 qui n'a pas
fonctionné.
\item $\square$ faire migrer le témoin positif.
\end{itemize}

\subsection{(2) PCR cible de l'ancre}
\label{sec:orgheadline15}
Ajouté le \textit{[2016-01-12 Tue 14:15]}

Chez les différents clones montrant des néomutations, on veut séquencer l'ancre,
pour vérifier que le taux de mutation dans la tract de conversion est bien
supérieur au taux de mutation dans une zone parfaitement homologue. L'hypothèse,
floue pour l'instant, est que le système de réparation des mésappariemments est
"saturé", et qu'il introduit des erreurs. (Ces mutations semblent biaisées vers
GC d'après les données dont on dispose pour l'instant.)

Le but est ici de réaliser une PCR simple sur quelques candidats, avant de
passer aux grands volumes, envoyés à séquencer. Les conditions de PCR sont
tirées du cahier de Florence, page 146, daté du \textit{[2015-06-25 Thu]}. 

\subsubsection{Candidats sélectionnés}
\label{sec:orgheadline11}
\begin{center}
\begin{tabular}{ll}
\toprule
strong & weak\\
\midrule
pS10 & pW14\\
pS24 & pW19\\
\bottomrule
\end{tabular}
\end{center}

\subsubsection{Programme PCR}
\label{sec:orgheadline12}
\begin{center}
\begin{tabular}{lrl}
\toprule
Temps & température & 30 cycles\\
\midrule
3' & 94 & \\
50'' & 94 & x\\
50'' & 55 & x\\
60'' & 72 & x\\
4' & 72 & \\
\bottomrule
\end{tabular}
\end{center}

\subsubsection{Volumes :}
\label{sec:orgheadline13}
\begin{center}
\begin{tabular}{lrr}
\toprule
 & 1-tube & 7-tubes\\
\midrule
ADN & 0.5 & 3.5\\
Taq & 0.5 & 3.5\\
amorce 1410 & 2.5 & 17.5\\
amorce 1411 & 2.5 & 17.5\\
MgCl2 & 1.5 & 10.5\\
dNTPs & 5.0 & 35.\\
Tampon 10X Taq & 5.0 & 35.\\
\midrule
eau qsp 50µL & 32.5 & 227.5\\
\bottomrule
\end{tabular}
\end{center}

\subsubsection{Contrôle :}
\label{sec:orgheadline14}

\begin{figure}[htb]
\centering
\includegraphics[width=4cm]{/Volumes/HDD/stage/img/20160113_00.JPG}
\caption{\label{fig:orgparagraph1}
Le contrôle de la PCR cible de l'ancre 100\% homologue. Le témoin positif de la PCR d'hier est censé être là également. }
\end{figure}

Voir figure \ref{fig:orgparagraph1}. 

\begin{itemize}
\item $\square$ Refaire migrer le témoin positif.
\end{itemize}

\subsection{(D) dilué amorces}
\label{sec:orgheadline16}
Ajouté le \textit{[2016-01-12 Tue 10:02]}

Dilué au 10\(^{\text{e}}\) (10 / 90µL eau aguettant) les amorces 1073, 1392, 1410 et 1411.

\subsection{(D) coulé gel}
\label{sec:orgheadline17}
Ajouté le \textit{[2016-01-12 Tue 16:01]}

Coulé deux grands gels 300mL agarose 1\% 26 puits.
\section{2016-01-13 Mercredi}
\label{sec:orgheadline29}
\subsection{(2) PCR cible des ancres}
\label{sec:orgheadline23}
Ajouté le \textit{[2016-01-13 Wed 09:36]}

La PCR d'hier ayant fonctionné, le but est de faire la même PCR dans les mêmes
conditions pour séquencer toutes les ancres des clones montrant des
néomutations. 

\subsubsection{candidats sélectionnés}
\label{sec:orgheadline19}
\begin{center}
\begin{tabular}{ll}
\toprule
strong & weak\\
\midrule
pS10 & pW14\\
pS24 & pW19\\
pS30 & pW2\\
pS39 & pW23\\
pS5 & pW35\\
pS54 & pW6\\
pS74 & pW81\\
pS82 & pW87\\
pS88 & pW93\\
\bottomrule
\end{tabular}
\end{center}

\begin{itemize}
\item erreur : trompé de clone, prélevé pW5 au lieu de pW2… gros malin.
\end{itemize}

\subsubsection{programme PCR}
\label{sec:orgheadline20}
\begin{center}
\begin{tabular}{lrl}
\toprule
Temps & température & 30 cycles\\
\midrule
3' & 94 & \\
50'' & 94 & x\\
50'' & 55 & x\\
60'' & 72 & x\\
4' & 72 & \\
\bottomrule
\end{tabular}
\end{center}

\subsubsection{volumes}
\label{sec:orgheadline21}
\begin{center}
\begin{tabular}{lrr}
\toprule
 & 1-tube & 19-tubes\\
\midrule
ADN & 0.5 & 9.5\\
Taq & 0.5 & 9.5\\
amorce 1410 & 2.5 & 47.5\\
amorce 1411 & 2.5 & 47.5\\
MgCl2 & 1.5 & 28.5\\
dNTPs & 5.0 & 95.\\
Tampon 10X Taq & 5.0 & 95.\\
\midrule
eau qsp 50µL & 32.5 & 617.5\\
\bottomrule
\end{tabular}
\end{center}

\subsubsection{contrôle}
\label{sec:orgheadline22}
Voir figure \ref{fig:orgparagraph2}. 

\begin{figure}[htb]
\centering
\includegraphics[width=\linewidth]{/Volumes/HDD/stage/img/20160113_01.JPG}
\caption{\label{fig:orgparagraph2}
La PCR cible des ancres semble avoir fonctionné de partout. Dans l'ordre, de pS10 à pS88, puis pW14 à pW93, T- et T+. Le T- est clean, le T+ trop fort. Penser à moins déposer pour le T+.}
\end{figure}

Tout a fonctionné a priori. Les produits de PCR sont conservés au -20°C dans le
portoir orange à couvercle. Je refais la PCR pour l'isolat pW2.

\subsection{(3) PCR cible des gènes synthétiques}
\label{sec:orgheadline28}
Il n'y a pas de bandes dans trois cas d'amplification, l'hypothèse la plus
parcimonieuse étant que la PCR n'a pas fonctionné… Je la refais donc pour ces
trois isolats là. lala. 

\subsubsection{candidats}
\label{sec:orgheadline24}
pW5\(_{\text{3}}\) -- pW5\(_{\text{4}}\) -- pS53\(_{\text{3}}\) 

\subsubsection{programme PCR}
\label{sec:orgheadline25}
\begin{center}
\begin{tabular}{lrl}
\toprule
Temps & température & 30 cycles\\
\midrule
3' & 94 & \\
50'' & 94 & x\\
50'' & 58 & x\\
50'' & 72 & x\\
4' & 72 & \\
\bottomrule
\end{tabular}
\end{center}

\subsubsection{volumes}
\label{sec:orgheadline26}
\begin{center}
\begin{tabular}{lrr}
\toprule
 & 1-tube & 6-tubes\\
\midrule
ADN & 0.5 & 3.\\
Taq & 0.5 & 3.\\
amorce 1410 & 2.5 & 15.\\
amorce 1411 & 2.5 & 15.\\
MgCl2 & 1.5 & 9.\\
dNTPs & 5.0 & 30.\\
Tampon 10X Taq & 5.0 & 30.\\
\midrule
eau qsp 50µL & 32.5 & 195.\\
\bottomrule
\end{tabular}
\end{center}

\subsubsection{contrôle}
\label{sec:orgheadline27}
Voir figure \ref{fig:orgparagraph3}. 
Absence d'amplification. Repartir des boîtes, resuspendre les colonies, et
refaire la PCR. 

\begin{figure}[htb]
\centering
\includegraphics[width=0.3\linewidth]{/Volumes/HDD/stage/img/20160113_02.JPG}
\caption{\label{fig:orgparagraph3}
Rien à l'horizon. Ladder tordu, puisque gel qui se tortille dans la cuve… Bien fait pour moi.}
\end{figure}

\section{2016-01-14 Jeudi}
\label{sec:orgheadline38}
\subsection{(3) PCR des trois clones récalcitrants}
\label{sec:orgheadline34}
La PCR d'hier semble montrer qu'il n'y a pas d'ADN dans le tube. Le témoin
positif, qui fonctionne bien avec le couple d'amorce 1410--1411 ne fonctionne
pas avec les amorces 1073--1392, pour une raison que je ne m'explique encore
pas. 

Je veux donc repartir d'une nouvelle suspension, et refaire (encore une fois) la
PCR sur ces trois colonies. Le but est d'avoir un plan d'expérience équilibré,
avec quatre colonies isolées par candidat choisi.
\subsubsection{suspension des clones}
\label{sec:orgheadline30}
Les trois colonies sélectionnées, pW5\(_{\text{3}}\) -- pW5\(_{\text{4}}\) -- pS53\(_{\text{3}}\), sont resuspendues
dans 20µL d'eau pure. 
\subsubsection{programme PCR}
\label{sec:orgheadline31}
\begin{center}
\begin{tabular}{lrl}
\toprule
Temps & température & 30 cycles\\
\midrule
3' & 94 & \\
50'' & 94 & x\\
50'' & 58 & x\\
50'' & 72 & x\\
4' & 72 & \\
\bottomrule
\end{tabular}
\end{center}

\subsubsection{volumes}
\label{sec:orgheadline32}
\begin{center}
\begin{tabular}{lrr}
\toprule
 & 1-tube & 6-tubes\\
\midrule
ADN & 0.5 & 3.\\
Taq & 0.5 & 3.\\
amorce 1410 & 2.5 & 15.\\
amorce 1411 & 2.5 & 15.\\
MgCl2 & 1.5 & 9.\\
dNTPs & 5.0 & 30.\\
Tampon 10X Taq & 5.0 & 30.\\
\midrule
eau qsp 50µL & 32.5 & 195.\\
\bottomrule
\end{tabular}
\end{center}

\subsubsection{contrôle}
\label{sec:orgheadline33}


\begin{figure}[htb]
\centering
\includegraphics[width=0.5\linewidth]{/Volumes/HDD/stage/img/20160114_00.JPG}
\caption{Les trois clones ont bien amplifié. Le T+ a toujours le même soucis, je suis pourtant parti d'un nouvel ADN génomique d'Acineto.}
\end{figure}
\subsection{(1) Contrôle des extractions plasmidiques}
\label{sec:orgheadline36}
Les extractions des plasmides porteurs des gènes synthétiques du
\textit{[2016-01-08 Fri] } sont contrôlées au nano-drop.

Résultat :
\begin{center}
\begin{tabular}{ll}
\toprule
plasmide & dosage\\
\midrule
WS & 8.5ng/µL\\
SW & 7.2ng/µL\\
\bottomrule
\end{tabular}
\end{center}

Il n'y a pratiquement rien. 

\subsubsection{relance des cultures}
\label{sec:orgheadline35}
J'ai sorti les cryotubes du -80°C, resuspendu les cultures dans 5mL de LB+Kan50,
incubation 24H à 37°C. Le but est de refaire les extractions demain
\textit{[2016-01-15 Fri] } avec plus de milieu, en culottant les 5mL de culture, et en
éluant dans un plus petit volume.

\subsection{(D) maintenance}
\label{sec:orgheadline37}
\begin{itemize}
\item nettoyé paillasse
\item aliquoté 10 \texttimes{} 1.5mL d'eau UP en eppendorf.
\end{itemize}
\end{document}