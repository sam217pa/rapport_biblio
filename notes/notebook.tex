% Created 2016-01-09 Sat 00:58
\documentclass[a4paper, justified]{tufte-handout}
\input{/Users/samuelbarreto/.templates/tufte-handout.tex}

\usepackage[utf8]{inputenc}
\usepackage[T1]{fontenc}
\usepackage{graphicx}
\usepackage{longtable}
\usepackage{float}
\usepackage{hyperref}
\usepackage{wrapfig}
\usepackage{rotating}
\usepackage[normalem]{ulem}
\usepackage{amsmath}
\usepackage{textcomp}
\usepackage{marvosym}
\usepackage{wasysym}
\usepackage{amssymb}
\usepackage[scaled=0.9]{zi4}
\usepackage[x11names, dvipsnames]{xcolor}
\usepackage[protrusion=true, expansion=alltext, tracking=true, kerning=true]{microtype}
\usepackage{siunitx}
\usepackage[frenchle, english]{babel}
\usepackage{booktabs}
\author{Samuel BARRETO}
\date{\today}
\title{}
\hypersetup{
 pdfauthor={Samuel BARRETO},
 pdftitle={},
 pdfkeywords={},
 pdfsubject={},
 pdfcreator={Emacs 24.5.1 (Org mode 8.3.2)}, 
 pdflang={English}}
\begin{document}

\tableofcontents


\section{Construction des recombinants \emph{Acinetobacter}}
\label{sec:orgheadline4}
\subsection{Resuspension des 5µg de gène synthétique}
\label{sec:orgheadline1}


Les gènes synthétiques ont été reçus lyophilisés, à
\SI{5}{\nano\g\per\milli\liter}\footnote{\textit{[2016-01-05 Tue]}}. Ils sont resuspendus
dans \SI{25}{\micro\liter}, à la concentration de
\SI{200}{\nano\g\per\milli\liter}. Les tubes sont aux positions A2 et A3 dans le
souchier au -20°C.

\subsection{Transformation dans E.coli}
\label{sec:orgheadline2}
On utilise le kit TOPO-TA, Invitrogen\footnote{\textit{[2016-01-05 Tue]}}
(\href{file:///Users/samuelbarreto/stage/doc/topota_cloning_kits.pdf}{voir le \emph{quick reference guide}}).
Deux tubes de cellules d'E.coli naturellement compétentes sont sorties du -80°C
et placés dans la glace. Ils sont marqués SW et WS, selon le gène alternant.
\SI{2}{\micro\liter} d'ADN de la solution de gène synthétique à
\SI{200}{\nano\gram\per\mol} (soit 0.2ng d'ADN) sont ajoutés, et laissés dans
la glace \si{30'} puis \si{30''} à 42°C. \si{250\micro\liter} de milieu SOC sont
ajoutés. La suspension est incubée \si{1\hour} à \si{37\celsius} sous agitation.

\SI{100}{\nano\mole} de chaque suspension sont étalés sur un milieu
LBA+kanamycine. Les milieux sont incubés \si{24\hour} à \si{37\celsius}. 


Les clones transformants sont isolés sur LBA +
kanamycine\footnote{\textit{[2016-01-06 Wed]}}. Les milieux sont incubés \si{24\hour} à
\si{37\celsius}.

Une colonie isolée de chaque plasmide SW et WS est suspendue dans
\si{5\milli\liter} de LB liquide + kanamycine, incubée \si{24\hour} à
\si{37\celsius}\footnote{\textit{[2016-01-07 Thu]}}.

\si{1\milli\liter} est prélevé pour les extractions
plasmidiques\footnotemark[4]{}, les 4 autres sont culottés, re-suspendus
dans \si{1\milli\liter} et placé en cryotubes\footnote{Annotation : WS E.coli TOPO
08-01 et SW E.coli TOPO 08-01}.

\subsection{Extractions plasmidiques}
\label{sec:orgheadline3}
Les extractions plasmidiques sont faites avec le kit \emph{NucleoSpin Plasmid}. 


\section{Séquençage des ancres}
\label{sec:orgheadline6}
Dans certaines séquences issues des résultats de Florence, on observe un taux de
mutation anormalement élevé au sein de la tract de conversion. Des mutations à
des positions de SNP inattendues sont apparues, indépendamment dans chaque
séquence. Ça n'est donc pas dû à une mutation dans le plasmide : on n'a pas de
mutation systématique. Pour vérifier que ce phénomène soit bien dû à la
conversion, on séquence les ancres 100\% homologues. L'hypothèse est que le taux
de mutation dans la tract de conversion est supérieur à celui dans l'ancre.

Dans le cas où le taux de mutation dans l'ancre ne serait pas significativement
supérieur à celui dans la tract de conversion (hypothèse nulle), il faudrait
chercher une autre explication à cette augmentation du taux de mutation due à la
recombinaison. 

\subsection{Sélection des candidats}
\label{sec:orgheadline5}
On sélectionne donc tous les candidats qui présentent des néo-mutations, d'après
l'analyse des données de SNP calling. 

\begin{margintable}
  \begin{center}
    \ttfamily
    \begin{tabular}{rr}
      \toprule
      \textbf{Strong} & \textbf{Weak} \\
      \midrule
      pS10 & pW14 \\
      pS24 & pW19 \\
      pS30 & pW2 \\
      pS39 & pW23 \\
      pS5  & pW35 \\
      pS54 & pW6 \\
      pS74 & pW81 \\
      pS82 & pW87 \\
      pS88 & pW93 \\
      \bottomrule
    \end{tabular}
  \end{center}
\end{margintable}

\section{Séquençage des clones avec des positions polymorphes}
\label{sec:orgheadline7}
Certains SNP dans les populations séquencées montrent des positions
hétérozygotes. Autrement dit, certains individus semblent avoir réparé à cette
position de SNP par un W, d'autre par un S. On a donc choisi de repiquer la
génération dont les séquences sont issues. 

\begin{figure}
  \includegraphics[width=\linewidth]{../seq_novembre/analysis/candidats_heterozygotes.pdf}
  \caption{Chaque ligne représente les variations de la qualité du base calling
    aux positions de SNP attendues, en fonction de la position sur le gène. Les
    clones représentés sont ceux qui montrent des diminutions de la qualité aux
    positions attendues. Les clones pW5, pW22, pW85 et pS53, pS67, pS88 ont été
    choisis pour le séquençage.}
\end{figure}

3 transformants W et 3 transformants S montrant des positions hétérozygotes sont
choisis et repiqués sur milieu LBA + kanamycine. Dans l'hypothèse où ce serait
dû à une absence de réparation, 50\% des clones de la génération suivante
devraient avoir l'allèle W, et 50\% l'allèle S.

La séquence du locus d'intérêt est amplifiée par PCR sur colonie. Pour chaque
transformant initial, 4 isolats sont amplifiés, pour un total de 24 séquences.  
\end{document}