\newpage
\setstretch{1.1}
\bibliography{references}
\newpage
\pagenumbering{roman}
\setcounter{page}{1}
\setcounter{lof}{1}
\section*{Annexes}

% \subsubsection*{Interférences entre biais et sélection}

% \setstretch{1.5}
% Le destin d'un allèle soumis à de la conversion génique biaisée est déterminée
% par deux paramètres principaux : $S$ l'intensité de la pression de sélection à
% laquelle elle est soumise, et $B$ l'intensité de l'action du biais de
% conversion. Ces deux paramètres dépendent de $Ne$ la taille efficace de la
% population. 
% \begin{align}
%   S &= N_e \times s \\
%   B &= N_e \times b \\
% \end{align}

% $N_e$ étant un facteur commun, le rapport entre les deux forces déterminantes
% n'est fonction que de $s$ et de $b$, l'intensité du biais. 

% \newpage
\subsubsection*{Correction des mésappariements : le modèle {\em E.coli}}
\begin{figure*}[htbp]
  \centering
  \includegraphics[width=0.6\linewidth]{img/mutslh.pdf}
  \caption*{{\bf La réparation des mésappariements par le complexe MutSLH.}
    \rmfamily Le dimère de MutS reconnaît l’anomalie structurelle de l’ADN causé
    par un mésappariement. Il recrute alors l’enzyme MutL, qui à son tour
    recrute l’endonucléase MutH. Celle-ci introduit une cassure sur l’un des
    brins. Elle est suivie d’une résection par une exonucléase, puis par la
    synthèse {\em via} une
    ADN polymérase, sur la base du brin intact.\\
    {\em Adapté de Molecular Biology of The Gene, Watson, 2012.}}
  \label{mutslh}
\end{figure*}

\vfill


