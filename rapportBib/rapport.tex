% Created 2015-12-01 Tue 18:36
\documentclass[11pt, oneside]{scrartcl}
\usepackage[utf8]{inputenc}
\usepackage[T1]{fontenc}
\usepackage{graphicx}
\usepackage{longtable}
\usepackage{float}
\usepackage{hyperref}
\usepackage{wrapfig}
\usepackage{rotating}
\usepackage[normalem]{ulem}
\usepackage{amsmath}
\usepackage{textcomp}
\usepackage{marvosym}
\usepackage{wasysym}
\usepackage{amssymb}
\usepackage[scaled=0.9]{zi4}
\usepackage[x11names, dvipsnames]{xcolor}
\usepackage[protrusion=true, expansion=alltext, tracking=true, kerning=true]{microtype}
\usepackage{siunitx}
%%
%% references
%%
\usepackage{csquotes}
\usepackage[super]{natbib}
\bibliographystyle{asm}
\usepackage[]{multicol}
%% todo
\usepackage[french]{todonotes}

%%
%% acronymes
%%
\usepackage[]{acro}
% probably a good idea for the nomenclature entries:
\acsetup{first-style=short}

% class `abbrev': abbreviations:
\DeclareAcronym{gbgc}{
  short = gBGC ,
  long  = GC--biased gene conversion : conversion génique biaisée vers GC ,
  class = abbrev
}
\DeclareAcronym{dsb}{
  short = DSB ,
  long  = Double strand break : cassures doubles brins ,
  class = abbrev
}

%%
%% table of content
%%
\setcounter{tocdepth}{2}
%% pas de titre
\deftocheading{toc}{\section*{Sommaire}}%
% \renewcommand{\contentsname}{Sommaire}


%%
%% typographie
%%
\providecommand\newthought[1]{%
   \addvspace{1.0\baselineskip plus 0.3ex minus 0.1ex}%
   \noindent\textsc{#1} % small caps text out
}

%% 
%% color
%% 
% \usepackage[usenames, dvipsnames]{xcolor}
% \setkomafont{section}{
%   \normalfont\huge\sffamily\bfseries\color{Cerulean}
%   {\llap{\colorbox{Cerulean}{\parbox{1.5cm}{\hfill\color{white}\thesection}}}}% label
% }
% \setkomafont{subsection}{\normalfont\large\itshape\sffamily\bfseries\color{TealBlue}}

\usepackage[]{titlesec}
%% section format
\titleformat{\section}%
  {\normalfont\huge\color{Cerulean}}% format applied to label+text
  {\llap{\colorbox{Cerulean}{\parbox{1.5cm}{\hfill\color{white}\thesection}}}}% label
  {1em}% horizontal separation between label and title body
  {}% before the title body
  []% after the title body

% subsection format
\titleformat{\subsection}%
  {\normalfont\large\itshape\color{TealBlue}}% format applied to label+text
  {\llap{\colorbox{TealBlue}{\parbox{1.5cm}{\hfill\color{white}\thesubsection}}}}% label
  {2em}% horizontal separation between label and title body
  {}% before the title body
  []% after the title body
\author{Samuel BARRETO}
\date{\today}
\title{La conversion génique vers GC : processus majeur d'évolution des génomes}
\hypersetup{
 pdfauthor={Samuel BARRETO},
 pdftitle={La conversion génique vers GC : processus majeur d'évolution des génomes},
 pdfkeywords={},
 pdfsubject={},
 pdfcreator={Emacs 24.5.1 (Org mode 8.3.2)}, 
 pdflang={English}}
\begin{document}

\pagenumbering{roman}
\tableofcontents
\newpage
\printacronyms[include-classes=abbrev,name=Abbréviations]
\newpage
\listoftodos
\newpage

\pagenumbering{arabic}
\setcounter{page}{1}

\section*{Introduction}
\label{sec:orgheadline1}
\addfig{%
  \centering
  \includegraphics[width=\linewidth]{img/holliday.pdf}
  \caption{\textbf{Le modèle classique de formation d'un hétéroduplex par
      invasion de brin.} \rmfamily%
    \setstretch{1.1} %
    Après la formation d'une cassure double brin en \texttt{a}, les extrémités
    $3'$ sont exposées par résection, en \texttt{b}. Le premier brin
    \crule[LightGray]{0.5cm}{0.15cm} porteur de l'allèle B envahit le brin
    \crule[Cyan]{0.5cm}{0.15cm} porteur de l'allèle b, formant une jonction de Holliday, en \texttt{c}.
    Localement, l'hétéroduplex ainsi formé montre des mésappariements, en
    \texttt{d}. \\
    {\em Adapté de Molecular Biology Of The Gene, Watson, 2012. } }
  \label{holliday}
}

La recombinaison homologue est un processus majeur d'évolution des génomes.
C'est un processus très conservé, des procaryotes aux
eucaryotes\cite{cromie_recombination_2001}. C'est un mécanisme clé de réparation
des lésions de l'ADN. Chez les eucaryotes, au cours de la méiose, des lésions
sont introduites, qui sont réparées par recombinaison homologue. Cette
réparation a deux fonctions : elle est requise pour la bonne conduite de la
séparation des chromosomes et induit un brassage des combinaisons d'allèles
bénéfique sur le plan évolutif\cite{webster_direct_2012}. Chez les procaryotes,
cette matrice peut être acquise par transfert horizontal de gène, \emph{via} les
prophages, la conjugaison ou la transformation naturelle. La recombinaison est
le moyen principal de transfert de gène entre procaryotes. Elle peut causer
l'acquisition de matériel génétique exogène, la perte de gènes ou le
remplacement d'allèles\cite{coupat-goutaland_ralstonia_2011}. Chez les bactéries
endosymbiotiques, la perte de la machinerie de réparation a causé leur
dégénérescence\cite{moran_genomics_2008}.

La recombinaison homologue est généralement abordée dans le cadre de ces deux
aspects positifs : la réparation des lésions et le brassage génétique qui en
découle. Elle possède cependant une face cachée : la conversion génique. Au
cours de la recombinaison homologue, des échanges d'information génétique de
courte portée ont lieu\cite{duret_biased_2009}. Ils sont dûs à la correction des
erreurs d'appariement dans les hétéroduplex formés ou à l'action de la
machinerie de réparation des lésions. Ils donnent lieu à des transmissions
non-réciproques d'information génétique entre les deux séquences recombinantes :
l'information génétique est transmise d'un brin vers l'autre, de façon
unidirectionnelle. Cette conversion est dite biaisée dès lors que l'un des
allèles a une plus forte probabilité de transmission, à l'échelle de la
population. Chez les eucaryotes, de nombreuses observations ont montré que la
conversion génique favorise les bases G ou C dans les cas d'hétérozygotie
AT/GC\cite{pessia_evidence_2012,mancera_high-resolution_2008,duret_impact_2008}.
Les allèles G ou C sont plus souvent transmis que les allèles A ou T. Il a été
montré que le biais de conversion génique en faveur de GC, ou \ac{gbgc}, a un
impact fort sur l'évolution des génomes : chez les mammifères, c'est le
principal déterminant de la structuration en GC des
génomes\cite{duret_impact_2008}.

Le contenu en GC est anormalement élevé dans les régions fortement
recombinantes. Ce mécanisme permet d'expliquer en partie les variations du taux
de GC, ou \ac{gc}, au sein d'un génome et entre génomes. L'explication de ces
variations suscite un débat entre les partisans des hypothèses adaptatives et
ceux des hypothèses neutres. Les premiers veulent que le contenu en GC soit un
trait sélectionné : il contribue en soi à la valeur adaptative de l'individu
\cite{hildebrand_evidence_2010}. Les seconds avancent qu'il est également
déterminé par des processus de dérive génétique et de mutation. Le \ac{gbgc} est
un autre processus neutre contribuant à déterminer le \ac{gc}. En fait, c'est la
contribution relative des trois processus à la détermination du taux de GC qui
fait débat.

De plus, le biais peut fixer des mutations faiblement délétères, si celles-ci
sont portées par les allèles G ou C du gène. Il contrecarre l'action de la
sélection naturelle\cite{galtier_gc-biased_2009, galtier_adaptation_2007}. Son
action laisse des traces génomiques équivalentes à celles de la sélection
naturelle : les tests classiques de détection confondent l'action de la
sélection avec celle du \ac{gbgc}\cite{ratnakumar_detecting_2010}.

Des preuves indirectes montrent que le gBGC est probablement à l'œuvre chez la
plupart des eucaryotes\cite{pessia_evidence_2012}. Récemment, l'hypothèse du
biais vers GC s'est étendue aux procaryotes\cite{lassalle_gc-content_2015}. Les
observations montrent que les régions fortement recombinantes ont un contenu en
GC significativement plus élevé. L'extension de cette hypothèse aux procaryotes
représente une avancée majeure : elle pourrait permettre d'expliquer l'évolution
du contenu en GC des populations bactériennes.

Nous verrons donc dans une première partie ce qu'est la conversion génique
biaisée et les mécanismes qui la sous-tendent; dans une deuxième, les hypothèses
permettant d'expliquer son existence ; et dans une dernière, ses conséquences
évolutives.

\section{Qu'est-ce que la conversion génique ?}
\label{sec:orgheadline8}

\addfig{%
  \centering
  \includegraphics[scale=0.7]{img/conversion.pdf}
  \caption{\textbf{Le modèle classique de réparation des cassures doubles brins
      par recombinaison homologue} \rmfamily%
    \setstretch{1.1} %
    Les cassures doubles brins sont suivies d'une résection dans le sens $5'
    \rightarrow 3'$. L'un des brins ainsi exposé cherche ensuite activement une
    séquence homologue. Au cours de l'invasion de brin, une boucle D se forme,
    ainsi qu'une jonction de Holliday. La synthèse d'ADN a lieu en utilisant la
    séquence intacte comme matrice. La résolution de ces structures peut passer
    par différentes voies. En \texttt{b}, le brin réparé doit être apparié à la
    l'extrémité du brin originel : c'est la {\em second-end capture.} Selon le
    mode de clivage des résolvases, le produit obtenu est non-crossover ou
    crossover (\texttt{d}). Chez les eucaryotes, la {\em dissolution} est une
    autre voie de résolutions des doubles jonctions de Hollidays (\texttt{e}).
    En \texttt{c}, la voie \ac{sdsa}, Synthesis-Dependent Strand-Annealing,
    implique une étape de dénaturation, puis de ré-appariement du brin
    envahisseur avec l'autre extrémité $3'$ de la cassure. La synthèse se
    poursuit et est suivie d'une étape de ligation. Dans tous les cas, des
    hétéroduplex sont formés, dès lors que les séquences
    appariées ne sont pas rigoureusement identiques. \\
    {\em Tiré de Chen {\em et al.}, 2007\cite{chen_gene_2007}.} }
  \label{recombinaison}
}

\subsection{Recombinaison homologue}
\label{sec:orgheadline5}

La réplication de l'ADN est une étape clé du cycle cellulaire. Elle implique la
formation de fourches de réplications. Les lésions subies par l'ADN en cours de
réplication empêchent le déplacement de ces fourches, et notamment les cassures
double brins. Ces cassures sont extrêmement cytotoxiques : elles causent l'arrêt
de la réplication, la perte de chromosomes, sont mutagènes et conduisent à la
mort cellulaire. La bonne conduite de la réplication nécessite de réparer ces
cassures, de façon non-mutagène. La recombinaison homologue est le moyen
préférentiel de réparation. Compte tenu de son importance dans le cycle
cellulaire, les mécanismes qui la sous-tendent sont ubiquitaires et extrêmement
conservés, des phages aux mammifères \cite{cromie_recombination_2001}.

\subsubsection{Mécanismes de recombinaison homologue}
\label{sec:orgheadline2}

Les cassures doubles brins, ou \ac{dsb}, sont générées par des agents mutagènes
ou sous contrôle génétique, notamment lors de la méiose. Après la formation des
\ac{dsb}, les extrémités \(5'\) de la lésion subissent une résection, délétion
locale d'une section d'ADN. Chez \emph{E.coli}, le complexe RecBCD est responsable de
cette digestion. Il a une activité d'hélicase, qui dissocie les deux brins
d'ADN, et de nucléase, qui dégrade localement les brins dissociés
\cite{dillingham_recbcd_2008}.

Le brin à l'extrémité \(3'\) envahit les brins complémentaires de la molécule
intacte. Ce brin recherche ensuite activement les zones d'homologie. Lorsqu'une
zone est déterminée, les brins sont échangés. L'ensemble de ce processus est
catalysé par la nucléo-protéine RecA \cite{chen_mechanism_2008} ou ses
homologues. C'est l'acteur moléculaire clé de la recombinaison homologue.

L'échange des brins entraîne la formation d'un hétéroduplex (figure
\ref{holliday}). L'intégrité de l'hétéroduplex est maintenue par une structure
appelée jonction de Holliday. Des mésappariements peuvent exister entre les
brins de l'hétéroduplex : deux brins homologues ne sont pas systématiquement
complémentaires. Leur correction peut donner lieu à une conversion génique.

La zone de résection est ensuite comblée par la synthèse d'ADN en utilisant le
brin homologue comme matrice. Enfin, les intermédiaires de recombinaisons sont
résolus par des résolvases qui clivent les jonctions de Holliday. La résolution
des intermédiaires de recombinaisons peut donner des produits dits crossovers ou
non-crossovers, entraînant respectivement l'échange des régions flanquantes ou
non \cite{mancera_high-resolution_2008} (figure \ref{recombinaison}). 

\begin{transition}
La réparation des cassures est la fonction principale et première de la
machinerie de recombinaison homologue. Cependant, les mécanismes en jeu sont le
lieu d'un brassage génétique, aussi bien lors de la méiose eucaryote que lors
des transferts de gène procaryotes \cite{redfield_bacteria_2001}.
\end{transition}

\subsubsection{La recombinaison méiotique : étape clé de la méiose}
\label{sec:orgheadline3}

Chez les eucaryotes, la méiose implique la formation de DSB, par les enzymes
Spo11, sous contrôle génétique rigoureux. Ils sont réparés par recombinaison
homologue \cite{chapman_playing_2012}. Cependant, la distribution des sites de
coupure est variable : il existe des hotspots de cassure, donc de recombinaison.
Par opposition, les coldspots sont des régions moins soumises que d'autres aux
cassures.

La réparation des \ac{dsb} par recombinaison homologue est requise pour
l'appariement et la ségrégation des chromosomes homologues au cours de la
méiose. Selon le mode de clivage des jonctions de Holliday par les résolvases,
des crossovers se forment entre les chromosomes parentaux. Ces crossovers
entraînent le brassage des allèles, un processus bénéfique sur le plan
évolutif\cite{webster_direct_2012}. En effet, il casse les liaisons entre
allèles : la sélection élimine alors plus efficacement les variants délétères et
promeut les variants bénéfiques.

\subsubsection{La recombinaison chez les procaryotes}
\label{sec:orgheadline4}

Étant donné la taille des populations bactériennes et les temps évolutifs en
jeu, la recombinaison a un impact majeur sur l'évolution procaryote
\cite{didelot_impact_2010}. C'est l'un des moteurs de transferts de gène.
Ceux-ci sont médiés soit par des vecteurs, les plasmides ou les phages, soit par
un état de compétence naturelle, \emph{via} l'acquisition passive ou active d'ADN
exogène. La principale fonction de la recombinaison homologue semble être la
réparation des lésions de l'ADN \cite{fall_horizontal_2007}, l'acquisition de
matériel génétique exogène est un effet secondaire des mécanismes de réparation
de l'ADN. Cet effet secondaire est bénéfique sur le plan évolutif dès lors que
le matériel acquis apporte un avantage sélectif à l'individu, ou s'il manipule
les caractères de l'hôte en faveur de sa
dissémination\cite{gogarten_horizontal_2005}.

\begin{transition}
Après la résolution des intermédiaires de recombinaison, des mésappariements
peuvent exister entre les différents brins. Leur correction entraîne une
conversion génique.
\end{transition}

\addfig{%
  \input{img/conversion_scheme.tex}
  \caption{\textbf{La conversion génique à ses différentes échelles.} \rmfamily%
    \setstretch{1.1}%
    À un locus \alg, dont les allèles sont \alA et \ala, la conversion génique
    entraîne soit la conversion de \ala par \alA, à gauche, soit l'inverse, à
    droite. À l'échelle d'un évènement de méiose, les gamètes obtenus sont soit
    \alA\alA\alA\ala, soit \alA\ala\ala\ala. À l'échelle de l'{\em ensemble de
      la gamétogénèse} d'un individu (en \texttt{a}), la conversion est biaisée
    si \alA\alA\alA\ala est plus souvent obtenu qu'\alA\ala\ala\ala. À l'échelle
    de la {\em population} (en \texttt{b}), la conversion est biaisée vers
    \alA si les individus dont la conversion est biaisée vers \alA sont plus
    fréquents. La conversion génique biaisée augmente la fréquence de l'allèle
    donneur dans le pool de gamète, donc sa probabilité de fixation dans la
    population. }
  \label{conversion}
}

\subsection{Conversion génique}
\label{sec:orgheadline6}

La conversion génique est l'échange non réciproque d'information
génétique\cite{chen_gene_2007}.

Considérons le cas de la transmission de l'allèle \(A\) et de son homologue \(a\),
au cours de la méiose. La méiose engendre deux gamètes \(A\) et deux gamètes \(a\).
Un évènement de conversion de gène à ce locus engendre soit un gamète \(A\) et
trois \(a\), soit un gamète \(A\) et trois \(a\) (figure \ref{conversion}). 

Au cours de la réparation des DSB, la conversion peut subvenir de deux façons.
i) L'allèle \(A\) est proche du site d'initiation de la cassure. Il fait partie de
la résection, l'allèle \(a\) est copié vers le brin réparé. Après la méiose, les
gamètes obtenus sont de type \(Aaaa\). ii) L'intermédiaire de recombinaison
présente un polymorphisme \(Aa\) sur l'un des hétéroduplex. La machinerie de
réparation des mésappariements --- \ac{mmr} --- les prend en charge. \(a\) est
alors converti en \(A\), ou réciproquement.

Chez \emph{E.coli}, la détection des mésappariements est effectuée par les dimères
des enzymes MutS. Les mésappariements sont reconnus par la distorsion qu'ils
causent à la structure de l'ADN. Les enzymes MutL et MutH sont alors recrutées.
Une cassure est introduite dans l'un des brins, suivie par une résection souvent
supérieure à \(1\) kb à proximité de la cassure. Une ADN polymérase utilise
ensuite le brin intact pour synthétiser la région complémentaire. Les eucaryotes
possèdent des protéines aux fonctions homologues, appelées respectivement MSH et
MLH pour MutS Homologs et MutL Homologs. Ce sont des composants de la voie
\ac{ner}, Nucleotide Excision Repair.

Au cours de la recombinaison, le système de \ac{mmr} est la voie préférentielle
de correction des mésappariements dans l'hétéroduplex. Néanmoins, la voie
\ac{ber}, Base Excision Repair, est une alternative à ce système. Elle entraîne
l'excision de l'une des bases du mésappariement, puis son remplacement par la
base complémentaire à l'autre. Les ADN glycosylases excisent les bases avec une
spécificité de substrat : chaque base A, T, C ou G a une ADN glycosylase
correspondante et spécifique.

Dans tous les cas, le génotype de la région --- ou de la base --- digérée est
converti par celui du brin intact. Le transfert a lieu entre séquences
homologues, qu'elles soient sur des chromatides sœurs, sur le même chromosome ou
sur des chromosomes différents \cite{chen_gene_2007}.

\begin{transition}
  En théorie, la conversion $a \mapsto A$ a lieu avec la même fréquence que
  celle de la conversion $A \mapsto a$. Cependant, dès lors qu'un allèle est
  plus souvent converti que l'autre, à l'échelle de la population, la conversion
  génique est biaisée (voire figure \ref{conversion}). Chez les eucaryotes, de
  nombreuses observations montrent que les mésappariements GA, GT, CA ou CT sont
  plus fréquemment corrigés en GC qu'en AT \cite{duret_biased_2009}.
\end{transition}

\addfig{% 
  \centering
  \includegraphics[width=0.5\linewidth]{img/cytosine.pdf}
  \caption{\textbf{La déamination spontanée des méthyl-cytosines.} \rmfamily
    \setstretch{1.1} La perte du groupement {\color{Red} amine} d'une cytosine
    méthylée génère une base naturelle de l'ADN : la thymine. Cette perte peut
    avoir lieu dans des conditions physiologiques normales. La réplication d'une
    telle erreur conduit à l'introduction d'un A sur le brin opposé, au lieu du
    G attendu. Le mécanisme de Base Excision Repair excise préférentiellement
    les thymines chez les vertébrés, probablement pour
    compenser la déamination spontanée des cytosines méthylées. \\
    {\em Adapté de Molecular Biology Of The Gene, Watson, 2012. } }
      \label{cytosine}
}
\subsection{Conversion génique biaisée vers GC}
\label{sec:orgheadline7}

Mancera \emph{et al} \cite{mancera_high-resolution_2008} ont génotypé l'ensemble des
quatre haplotypes --- les tétrades --- résultants des produits de méiose de 46
levures, à haute résolution. Ils montrent qu'1\% du génome de chaque produit de
méiose est soumis à de la conversion génique. Ces régions montrent une
transmission biaisée en faveur des allèles G ou C. Ils sont transmis avec une
probabilité 1,3\% plus élevée qu'attendu sous l'hypothèse d'une transmission
mendélienne \cite{mancera_high-resolution_2008}. Ce biais, bien que faible, peut
affecter très fortement la probabilité de fixation des allèles GC dès lors que
la taille de la population est grande\cite{nagylaki_evolution_1983}. 

Chez la levure, le \ac{gbgc} est associé spécifiquement aux produits de
recombinaisons entraînant des crossovers \cite{lesecque_gc-biased_2013}. Il est
également associé aux évènements de conversion simple --- par opposition aux
évènements complexes. Lors d'un évènement de conversion simple, le même brin est
le donneur de la conversion sur l'ensemble de la région convertie. Lors d'un
évènement complexe, les deux brins de l'hétéroduplex peuvent être donneur. 

\begin{transition}
  Les causes moléculaires de l'existence d'un tel mécanisme suscitent beaucoup
  d'interrogations. Différentes hypothèses ont été avancées : elles font l'objet
  de la partie suivante. 
\end{transition}

\section{Quelles hypothèses pour expliquer le gBGC ?}
\label{sec:orgheadline11}
\subsubsection*{Mécanismes moléculaires ?}
\label{sec:orgheadline9}
\addcontentsline{toc}{subsection}{Mécanismes moléculaires}

Les mécanismes responsables du gBGC, ses causes proximales, n'ont pas encore été
formellement identifiées à ce jour. Différentes hypothèses ont cependant été
avancées, l'hypothèse privilégiée étant un biais causé par la machinerie de
réparation. 

Chez les mammifères, le mécanisme de BER dans les cellules en mitose est
fortement biaisée vers G ou C. Les ADN glycosylases de cette voie ciblent
spécifiquement les bases thymines, probablement pour compenser l'hypermutabilité
des cytosines\cite{brown_specific_1987} (figure \ref{cytosine}). L'intervention
du BER dans la réparation des mésappariements au cours de la méiose n'a
cependant jamais été démontré. 

De plus, chez la levure, l'hypothèse de l'intervention du \ac{ber} a été
exclue\cite{lesecque_gc-biased_2013}. En effet, étant donné la courte portée du
BER, les traces de conversion obtenues devraient être complexes, avec une
alternance des génotypes parentaux sur de courtes échelles. Pourtant, le biais
de conversion n'est observé que dans les traces simples : le brin donneur est le
même sur l'ensemble de la région convertie.

Deux modèles alternatifs ont été proposés\cite{lesecque_gc-biased_2013} : i) le
modèle de rejet de brin, et ii) le modèle du \ac{mmr} biaisé. Le modèle de rejet
de brin intervient au moment de la recherche d'homologie par le complexe
RecA-ADN simple brin : si un brin riche en AT est moins souvent rejeté que son
homologue riche en GC, la conversion a plus souvent lieu du brin riche en GC
vers le brin riche en AT. Ce qui causerait une sur-transmission de GC. 

Le modèle du \ac{mmr} biaisé dépend du choix de brin matrice pour la réparation
des mésappariements. Chez les eucaryotes, le choix du brin matrice dépend
usuellement de la présence d'une cassure sur l'un des brins. Lorsqu'un
mésappariement est détecté par MSH, il recrute MLH. Le complexe ainsi formé
parcourt la région lésée, jusqu'à rencontrer une cassure. Il y recrute une
exonucléase responsable de la dégradation du brin contenant la cassure. Si des
cassures sont présentes sur les deux brins, la machinerie est face à un choix :
elle peut choisir le brin matrice. Le modèle du \ac{mmr} biaisé avance que le
choix favoriserait la dégradation du brin porteur d'une cassure proche d'une
base A ou T. Le brin porteur de G ou C serait alors plus souvent le donneur de
l'évènement de conversion. 

\subsubsection*{Un processus sélectionné pour compenser la mutation ?}
\label{sec:orgheadline10}
\addcontentsline{toc}{subsection}{Compenser la mutation ?} 

La raison d'être évolutive du gBGC est encore mystérieuse à ce jour. La mutation
étant universellement biaisée vers AT
\cite{lynch_rate_2010,hershberg_evidence_2010}, le \ac{gbgc} pourrait avoir été
sélectionné pour contrecarrer les effets de ce biais mutationnel
\cite{marais_biased_2003, birdsell_integrating_2002}. Le gBGC permettrait de
``guérir'' les mutations vers AT par recombinaison homologue. Il est également
possible que le gBGC soit dû à des mécanismes de réparation mitotiques, dont
l'action biaisée vers GC est conservée au cours de la recombinaison homologue
\cite{duret_biased_2009}.

Il est possible que le biais de conversion ne soit que le ``pendentif'' de l'arc
de la recombinaison : la sélection agit sur la réparation de l'ADN, de façon à
en conserver le fonctionnement, quitte à s'accomoder de sa face cachée, la
conversion génique biaisée (voir annexe \ref{sub:arcs}).

\section{Quelles sont les conséquences du gBGC ?}
\label{sec:orgheadline17}

 \addfig{%
  \centering
  \includegraphics[width=0.7\linewidth]{img/isochores.pdf}
  \caption{\textbf{Les isochores : des variations de \ac{gc} à grande
      échelle.}\rmfamily \setstretch{1.1} Est représentée ici la distribution du
    taux de GC sur le chromosome humain 6. Des régions relativement homogènes en
    taux de GC se distinguent. Leur distribution est très variable sur une
    échelle de $4$Mb. Le contenu en GC est corrélé à un grand nombre d'autres
    facteurs, tels que la densité de gène, le taux de transcription ou encore la
    vitesse de réplication. \\ {\em Tiré de Eyre-Walker \& Hurst,
      2001\cite{eyre-walker_evolution_2001}} }
  \label{isochores}

  \centering
  \includegraphics[width=0.7\linewidth]{img/pessia.pdf}
  \caption{\textbf{Un gBGC universel ? Corrélation entre le taux de
      recombinaison et le contenu en GC chez les eucaryotes. } \rmfamily%
    \setstretch{1.1}%
    Parmi 36 espèces tirées des groupes eucaryotes majeurs, Pessia et
    collaborateurs ont cherché à déterminer la relation entre \ac{gc} et taux de
    recombinaison. Les \tikzcircle[PineGreen, fill=PineGreen]{3pt} et
    \tikzcircle[PineGreen, fill=White]{3pt} indiquent une corrélation positive
    entre le taux de GC et le taux de recombinaison local. Les deux
    \tikzcircle[Red, fill=Red]{3pt} indiquent les corrélations négatives
    non-compatibles avec l'hypothèses gBGC. Cette étude semble montrer
    que le gBGC est un mécanisme universel chez les eucaryotes. \\
    {\em Tiré de Pessia {\em et al.}, 2012\cite{pessia_evidence_2012}.} }
   \label{pessia}
}

\subsection{Le gBGC structure le contenu en GC}
\label{sec:orgheadline14}
Les bases C et G sont liées par trois liaisons hydrogènes : elles sont plus
stables que les liaisons doubles entre A et T. Certains auteurs pensent qu'en
soi, le taux de GC est un trait adaptatif : à l'échelle du génome, un contenu en
GC supérieur en augmenterait la stabilité. Ce modèle rencontre néanmoins de
nombreuses difficultés, chez les eucaryotes comme les procaryotes. La conversion
biaisée vers GC a été proposée comme modèle alternatif expliquant les variations
de \ac{gc}, au sein d'un génome et entre génomes.

\subsubsection{Le contenu GC des génomes mammifères et la théorie des isochores}
\label{sec:orgheadline12}
Les mammifères montrent des variations intragénomiques de grande échelle en taux
de GC\cite{eyre-walker_evolution_2001} ( \(>\) 100kb ). Ces régions relativement
homogènes en taux de GC ont été baptisées isochores (figure \ref{isochores}).
Leur origine fait débat : est-ce un trait sélectionné ou une conséquence
évolutive des patrons de mutations ?

Le modèle sélectionniste se heurte au fait que les variations du GC affectent
les sites fonctionnels comme neutres. En fait, l'évolution des isochores résulte
de l'accumulation de mutations. Il faudrait donc un avantage sélectif
significatif à l'acquisition d'une mutation ponctuelle vers G ou C, dans un
isochore de plus de 100kb.

Le \ac{gbgc} a été proposé pour expliquer l'apparition et le maintien des isochores
riches en GC\cite{duret_new_2006}. Un argument fort de l'hypothèse \ac{gbgc} est que
les zones fortement recombinantes ont un \ac{gc} supérieur. C'est le cas chez
l'Homme\cite{duret_impact_2008, berglund_hotspots_2009}. L'apparition et la
disparition successive de points chauds de recombinaison explique la succession
des épisodes de \ac{gbgc} : il conditionne le contenu en GC local, permettant
d'expliquer la structuration des isochores riches en GC. 

La taille des chromosomes a un impact fort sur le \ac{gc} : le taux de
recombinaison à l'échelle de la Mb est fortement corrélé à la taille du
chromosome, chez le poulet et l'Homme\cite{kaback_chromosome_1999}. Autrement
dit, les grands chromosomes recombinent peu, les petits beaucoup.

Cette corrélation entre le taux de recombinaison et le contenu en GC local a
également été observée dans la plupart des taxons
eucaryotes\cite{pessia_evidence_2012} (figure \ref{pessia}).

\begin{transition}
  Ainsi, chez les mammifères, le contenu en GC est déterminé par la
  recombinaison : elle augmente la probabilité de fixation des mutations vers
  GC. Elle a pour impact de structurer localement le \ac{gc} au gré des épisodes
  de points chauds de recombinaisons. De nombreuses preuves indirectes attestent
  de l'existence du \ac{gbgc} chez les eucaryotes. Qu'en est-il chez les
  procaryotes ?
\end{transition}


\addfig{%
  \centering
  \includegraphics[scale=1.3]{img/lassalle.pdf}
  \caption{\textbf{Le gBGC chez les procaryotes ? Effet de la recombinaison sur
      le contenu en GC du \emph{core genome.}} \rmfamily%
    \setstretch{1.1} %
    La différence entre le contenu en GC des gènes recombinants et des gènes
    non-recombinants est mesurée sur l'ensemble de la séquence codante
    ( \lassalleFonce ) et sur la troisième position de codon uniquement
    ( \lassalleClair ). La troisième position est moins soumise à la sélection :
    les mutations peuvent être synonymes. Le taux de GC des gènes recombinants
    est significativement supérieur à celui des non-recombinants, \emph{a fortiori}
    lorsqu'on considère les positions les moins contraintes par la sélection. \\
    {\em Tiré de Lassalle {\em et al.}, 2015 \cite{lassalle_gc-content_2015}.}
  }
  \label{lassalle}
}

\subsubsection{Un \ac{gbgc} procaryote ?}
\label{sec:orgheadline13}
Le taux moyen de GC chez les procaryotes est extrêmement diversifié : il varie
de 14 à 75\% selon les espèces. Certains auteurs y voient une adaptation aux
conditions environnementales\cite{foerstner_environments_2005}. Cependant, les
pressions de sélection associées restent mystérieuses. Le modèle classique
considère que les variations de \ac{gc} sont essentiellement déterminées par la
mutation, qui est biaisée vers
AT\cite{hershberg_evidence_2010,sueoka_directional_1988}.

Cependant, de récents travaux ont démontré un biais de fixation des allèles G ou
C\cite{hildebrand_evidence_2010, hershberg_evidence_2010}. Deux hypothèses
peuvent expliquer ce biais de fixation : i) La sélection favoriserait
ponctuellement la fixation d'un G ou C --- l'hypothèse étant peu crédible --- ,
ou ii) le gBGC. Il a été démontré que les gènes recombinants ont un taux de GC
supérieur aux non-recombinants\cite{lassalle_gc-content_2015}, chez 21 espèces
bactériennes. La troisième position des codons est d'autant plus affectée
qu'elle est moins soumise à la sélection. Le code génétique étant redondant, une
mutation en troisième position ne change pas nécessairement l'acide aminé : la
mutation est synonyme. Les régions intergéniques flanquantes des gènes
recombinants ont également un \ac{gc} supérieur à celles des régions flanquantes
des gènes non-recombinants. C'est un patron attendu sous l'hypothèse \ac{gbgc}.
Cette corrélation entre taux de recombinaison et contenu en GC est similaire
quantitativement à celle observée chez l'Homme\cite{lassalle_gc-content_2015}.

%
\addfig{%
  \includegraphics[width=\linewidth]{img/pollard.pdf}
  \caption{\textbf{Le gBGC a-t-il interféré avec la sélection pour le
      développement des régions corticales humaines ? } \rmfamily%
      \setstretch{1.1} Pollard et collaborateurs ont analysé les régions
    non-codantes dont la vitesse d'évolution a augmenté uniquement dans la
    lignée humaine : les \ac{har}, Human Accelerated Regions. Parmi ces régions,
    la région HAR1 est particulièrement intéressante : elle comprend un gène
    codant pour un ARN régulateur exprimé au cours du développement des régions
    corticales. De façon surprenante, les 18 mutations spécifiques à l'Homme
    sont de type AT $\rightarrow$ GC (en \texttt{a}). Ces changements ont une
    influence sur la structure de l'ARN, représenté en \texttt{b} : l'hélice D
    est plus longue chez l'Homme que chez le Chimpanzé (en \texttt{c}).
    Autrement dit, le biais vers GC aurait pu influencer la divergence entre
    l'Homme et le Chimpanzé. Un ARN non-codant extrêmement conservé du Poulet au
    Chimpanzé a brutalement accéléré dans la lignée humaine. La sélection et le
    gBGC ont pu agir de concert pour altérer la structure de cet ARN. \\
    {\em Adapté de Pollard {\em et al.}, 2006 \cite{pollard_rna_2006}.} }
  \label{pollard}
}
%

\begin{transition}
  Le \ac{gbgc} explique donc en partie la structuration en GC des génomes
  mammifères, eucaryotes et probablement procaryote. Il augmente la probabilité
  de fixation des allèles G ou C : il peut même s'opposer à la sélection
  naturelle si cette mutation est faiblement délétère.  
\end{transition}


\subsection{Le gBGC interfère avec la sélection}
\label{sec:orgheadline15}
La conversion génique affecte la probabilité de fixation d'un allèle de façon
similaire à la sélection \cite{nagylaki_evolution_1983}. Si un allèle faiblement
délétère est porté par une mutation AT \(\rightarrow\) GC, le \ac{gbgc} peut
entraîner sa fixation dans la population. À l'inverse, il peut empêcher la
fixation d'une mutation GC \(\rightarrow\) AT. Il contrecarre les effets de la
sélection naturelle.

Galtier et collaborateurs ont analysé la séquence des protéines de primates qui
montrent un taux d'évolution plus rapide depuis la divergence avec les macaques
\cite{galtier_gc-biased_2009}. Cette accélération de la vitesse de substitution
dans les séquences codantes peut \emph{a priori} être due à un changement de fonction
adaptatif. Cependant, les séquences analysées montrent un excès significatif
de mutations AT \(\rightarrow\) GC. De plus, ces mutations sont significativement
plus souvent non-synonymes : elles changent l'acide aminé. En clair, des régions
auparavant conservées ont subi un ou plusieurs épisodes de gBGC, qui ont
entraîné deux choses : i) le \ac{gc} local a augmenté, et ii) la fonction des
régions a changé (figure \ref{pollard}).
\\

\subsection{Le gBGC fausse les tests de sélection}
\label{sec:orgheadline16}
Les tests de sélection reposent classiquement sur deux principes généraux
\cite{hurst_genetics_2009} : i) les régions fonctionnelles tendent à être
conservées : elles évoluent lentement, par rapport à la vitesse d'évolution aux
sites neutres ; et ii) les régions qui ont évolué rapidement ont subi un
changement de fonction adaptatif. L'approche privilégiée pour détecter les
régions influencées par la sélection consiste à comparer les génomes, puis à
identifier les régions qui évoluent rapidement, sur une branche particulière de
l'arbre phylogénétique obtenu\cite{ratnakumar_detecting_2010}.

Le gBGC a cependant une empreinte sur les séquences similaire à celle de la
sélection dirigée : il peut brouiller les tests de sélection. L'action de la
sélection naturelle est alors confondue avec celle d'un processus neutre voir
mal-adaptatif. Typiquement, le ratio \(\nicefrac{d_N}{d_S}\), qui résume le
rapport entre le taux de substitutions non-synonymes \(d_N\) et synonymes \(d_S\),
est supérieur à 1 lorsque la protéine est sous sélection positive, en faveur du
changement de fonction. Ratnakumar et collaborateurs ont estimé qu'environ 20\%
des régions avec un ratio \(\nicefrac{d_N}{d_S}\) élevé pourraient avoir été sous
l'influence du gBGC : le gBGC brouille les traces de la sélection naturelle.
\section*{Conclusion}
\label{sec:orgheadline18}
La conversion génique biaisée vers GC est un mécanisme potentiellement
universel, qui affecte localement le taux de GC des régions recombinantes. Bien
que généralement faible, le biais peut avoir un impact fort sur la fixation d'un
allèle, si la taille efficace de la population est grande. Il contribue à
structurer le taux de GC des génomes, peut contrecarrer la sélection, et pose
des problèmes de détection de celle-ci. De nombreuses interrogations restent en
suspens. Les mécanismes qui le sous-tendent sont encore mystérieux, de même que
sa raison d'être évolutive. L'extension récente de l'hypothèse gBGC aux
procaryotes reste à confirmer expérimentalement. Elle constitue néanmoins un
terrain d'exploration nouveau, qui pourrait permettre d'étudier plus avant la
machinerie moléculaire responsable d'un phénomène \emph{a priori} anodin, mais dont
les conséquences sont nombreuses. 
\newpage
\setstretch{1.1}
\bibliography{references}
\newpage
\pagenumbering{roman}
\setcounter{page}{1}

\end{document}