% Created 2015-11-22 Sun 20:30
\documentclass[11pt, oneside]{scrartcl}
\usepackage[utf8]{inputenc}
\usepackage[T1]{fontenc}
\usepackage{graphicx}
\usepackage{longtable}
\usepackage{float}
\usepackage{hyperref}
\usepackage{wrapfig}
\usepackage{rotating}
\usepackage[normalem]{ulem}
\usepackage{amsmath}
\usepackage{textcomp}
\usepackage{marvosym}
\usepackage{wasysym}
\usepackage{amssymb}
\usepackage[scaled=0.9]{zi4}
\usepackage[usenames, dvipsnames]{xcolor}
\usepackage[protrusion=true, expansion=alltext, tracking=true, kerning=true]{microtype}
\usepackage{siunitx}
%%
%% references
%%
\usepackage{csquotes}
\usepackage[super]{natbib}
\bibliographystyle{asm}
\usepackage[]{multicol}
%% todo
\usepackage[french]{todonotes}

%%
%% acronymes
%%
\usepackage[]{acro}
% probably a good idea for the nomenclature entries:
\acsetup{first-style=short}

% class `abbrev': abbreviations:
\DeclareAcronym{gbgc}{
  short = gBGC ,
  long  = GC--biased gene conversion : conversion génique biaisée vers GC ,
  class = abbrev
}
\DeclareAcronym{dsb}{
  short = DSB ,
  long  = Double strand break : cassures doubles brins ,
  class = abbrev
}

%%
%% table of content
%%
\setcounter{tocdepth}{2}
%% pas de titre
\deftocheading{toc}{\section*{Sommaire}}%
% \renewcommand{\contentsname}{Sommaire}


%%
%% typographie
%%
\providecommand\newthought[1]{%
   \addvspace{1.0\baselineskip plus 0.3ex minus 0.1ex}%
   \noindent\textsc{#1} % small caps text out
}

%% 
%% color
%% 
% \usepackage[usenames, dvipsnames]{xcolor}
% \setkomafont{section}{
%   \normalfont\huge\sffamily\bfseries\color{Cerulean}
%   {\llap{\colorbox{Cerulean}{\parbox{1.5cm}{\hfill\color{white}\thesection}}}}% label
% }
% \setkomafont{subsection}{\normalfont\large\itshape\sffamily\bfseries\color{TealBlue}}

\usepackage[]{titlesec}
%% section format
\titleformat{\section}%
  {\normalfont\huge\color{Cerulean}}% format applied to label+text
  {\llap{\colorbox{Cerulean}{\parbox{1.5cm}{\hfill\color{white}\thesection}}}}% label
  {1em}% horizontal separation between label and title body
  {}% before the title body
  []% after the title body

% subsection format
\titleformat{\subsection}%
  {\normalfont\large\itshape\color{TealBlue}}% format applied to label+text
  {\llap{\colorbox{TealBlue}{\parbox{1.5cm}{\hfill\color{white}\thesubsection}}}}% label
  {2em}% horizontal separation between label and title body
  {}% before the title body
  []% after the title body
\author{Samuel Barreto}
\date{\today}
\title{La conversion génique vers GC : processus majeur d'évolution des génomes}
\hypersetup{
 pdfauthor={Samuel Barreto},
 pdftitle={La conversion génique vers GC : processus majeur d'évolution des génomes},
 pdfkeywords={},
 pdfsubject={},
 pdfcreator={Emacs 24.5.1 (Org mode 8.3.2)}, 
 pdflang={English}}
\begin{document}

\pagenumbering{roman}
\tableofcontents
\newpage
\printacronyms[include-classes=abbrev,name=Abbréviations]
\newpage
\listoftodos
\newpage

\pagenumbering{arabic}
\setcounter{page}{1}

\section*{Introduction}
\label{sec:orgheadline1}
La recombinaison homologue est un processus majeur d'évolution des génomes.
C'est un processus très conservé, des procaryotes aux
eucaryotes\cite{cromie_recombination_2001}. C'est le processus clé de réparation
des lésions de l'ADN. Elles sont réparées sur la base d'une matrice d'ADN
homologue, donnant lieu à la formation transitoire d'un \emph{hétéroduplex}. Un
hétéroduplex est un appariement entre deux brins homologues, qui ne sont pas
nécessairement rigoureusement complémentaires. Chez les eucaryotes, la matrice
est typiquement la région homologue correspondante, sur la chromatide sœur. Les
crossover méiotiques sont générés sous contrôle génétique et impliquent une
étape obligatoire de recombinaison\cite{mancera_high-resolution_2008}. Chez les
procaryotes, cette matrice peut être acquise par transfert horizontal. La
recombinaison est le moyen principal de transfert de gène entre procaryotes
\cite{vos_rates_2015}.

\newthought{Au voisinage} de la région lésée --- puis réparée --- , des échanges
d'information génétique de petite échelle existent\cite{duret_biased_2009}. Ils
sont dûs à la correction des erreurs d'appariement dans l'hétéroduplex. Ce sont
des événements de \emph{conversion génique}. Ils donnent lieu à des transmissions
non-mendéliennes d'allèles : l'information génétique est transmise d'un brin
vers l'autre, de façon unidirectionnelle. Cette conversion est \emph{biaisée} dès
lors que l'un des allèles a une plus forte probabilité de transmission, à
l'échelle de la population. Chez les eucaryotes, de nombreuses observations
montrent que la conversion est biaisée \emph{vers les bases G ou C}
\cite{pessia_evidence_2012, mancera_high-resolution_2008, duret_impact_2008}.
Les allèles G ou C sont plus souvent transmis que les allèles A ou T. Ce biais
de conversion génique en faveur de GC --- \ac{gbgc} --- a un impact fort sur
l'évolution des génomes : il conditionne la \emph{structuration} en contenu GC des
génomes\cite{duret_impact_2008} ; il \emph{interfère} avec la sélection
naturelle\cite{galtier_gc-biased_2009} ; et, d'un point de vue pratique, il
\emph{brouille les tests} de sélection naturelle\cite{ratnakumar_detecting_2010}.

Dans les régions fortement recombinantes, le contenu en GC est anormalement
élevé, lorsque le biais est observé\cite{duret_impact_2008}. Ce mécanisme permet
d'expliquer en partie les variations en \ac{gc}, au sein d'un génome et entre
génomes. L'explication de ces variations suscite un débat entre les partisans
des hypothèses adaptatives, et ceux des hypothèses neutres. Les hypothèses
\emph{adaptatives} veulent que le contenu en GC soit un trait sélectionné : il
contribue en soi à la valeur adaptative de l'individu
\cite{hildebrand_evidence_2010}. Les hypothèses \emph{neutres} avancent qu'il est
également déterminé par des processus de dérive génétique et de mutation. Le
\ac{gbgc} est un autre processus neutre contribuant à déterminer le \ac{gc}. En
fait, c'est la contribution \emph{relative} des trois processus à la détermination du
\ac{gc} qui fait débat.

Le biais peut également permettre de fixer des mutations délétères, si celles-ci
sont portées par les allèles G ou C du gène. Il contrecarre l'action de la
sélection naturelle\cite{galtier_gc-biased_2009, galtier_adaptation_2007}.

Enfin, l'action du biais laisse des traces génomiques équivalentes à celles de
la sélection naturelle : les tests classiques de détection confondent l'action
de la sélection avec celle du \ac{gbgc}\cite{ratnakumar_detecting_2010}.

\newthought{Récemment,} l'hypothèse du biais vers GC s'est étendue aux
procaryotes\cite{lassalle_gc-content_2015}. Des observations montrent que les
régions fortement recombinantes ont un contenu en GC significativement plus
élevé. L'extension de cette hypothèse aux procaryotes représente une avancée
majeure : elle pourrait permettre d'expliquer l'évolution du contenu en GC des
populations bactériennes.

Nous verrons donc, dans une première partie, ce qu'est la conversion génique
biaisée et les mécanismes qui la sous-tendent ; dans une deuxième, ses
conséquences évolutives ; et dans une dernière, les hypothèses permettant
d'expliquer son existence.

\section{Qu'est-ce que la conversion génique ?}
\label{sec:orgheadline8}

La conversion génique est un mécanisme qui intervient au cours de la réparation
de l'ADN. Lors d'une cassure double brin --- \ac{dsb} --- , la lésion est réparée par
recombinaison homologue. Les mésappariements induits peuvent être corrigés,
donnant lieu à une transmission d'allèle non-mendélienne \cite{chen_mechanism_2008}. 

\addfig{%
  \centering
  \includegraphics[width=\linewidth]{img/conversion.pdf}
  \caption{\textbf{Recombinaison homologue} 
  }
  \label{recombinaison}
}

\subsection{Recombinaison homologue}
\label{sec:orgheadline6}

La réplication de l'ADN est une étape clé du cycle cellulaire. Elle implique la
formation de fourches de réplications. Les lésions subies par l'ADN en cours de
réplication empêchent le déplacement de ces fourches, et notamment les cassures
double brins. Ces \ac{dsb} sont extrêmement cytotoxiques : ils causent l'arrêt
de la réplication, la perte de chromosomes, sont mutagènes et conduisent à la
mort cellulaire \cite{watson_molecular_2014}. La bonne conduite de la
réplication nécessite de réparer ces cassures, de façon non-mutagène. La
recombinaison homologue est le moyen préférentiel de réparation
\cite{lusetti_bacterial_2002}. Compte tenu de son importance dans le cycle
cellulaire, les mécanismes qui la sous-tendent sont ubiquitaires et extrêmement
conservés, des phages aux mammifères \cite{cromie_recombination_2001}.

\subsubsection{Mécanismes de recombinaison homologue}
\label{sec:orgheadline3}

La recombinaison se déroule en plusieurs étapes. 

Les cassures doubles brins --- \ac{dsb} --- sont générées par des processus
mutagènes ou sous contrôle génétique, notamment lors de la méiose. 

Après la formation des \ac{dsb}, les extrémités \(5'\) de la lésion subissent une
\emph{résection}, délétion locale d'une section d'ADN. Chez \emph{E.coli}, le complexe
RecBCD est responsable de cette digestion. Il a une activité d'hélicase : il
dissocie les deux brins d'ADN ; et de nucléase : il digère localement les brins
dissociés \cite{dillingham_recbcd_2008}.

Le brin à l'extrémité \(3'\) \emph{envahit} les brins complémentaires de la molécule
intacte. Ce brin recherche ensuite activement les zones d'homologie. Lorsqu'une
zone est déterminée, les brins sont échangés. L'ensemble de ce processus est
catalysé par la nucléo-protéine RecA \cite{chen_mechanism_2008}. C'est l'acteur
moléculaire clé de la recombinaison homologue. Chez les eucaryotes, l'homologue
fonctionnel de RecA est le complexe Dmc1 \cite{bishop_dmc1:_1992}.

L'échange des brins entraîne la formation d'un hétéroduplex. L'intégrité de
l'hétéroduplex est maintenue par une structure appelée \emph{jonction de Holliday}.
Des mésappariements peuvent exister entre les brins de l'hétéroduplex : deux
brins homologues ne sont pas systématiquement complémentaires. Une fois résolus,
ces mésappariements peuvent être le lieu d'une conversion génique (voir
\ref{sec:orgheadline2}).

La zone de résection est ensuite comblée par la \emph{synthèse} d'ADN en utilisant le
brin homologue comme matrice. 

Enfin, les intermédiaires de recombinaisons sont \emph{résolus}, par le clivage
aléatoire des jonctions de Holliday. Ce clivage est catalysée par des
\emph{résolvases}, telles que RuvC \cite{gorecka_crystal_2013}. 

La résolution des intermédiaires de recombinaisons peut donner des produits dits
crossovers ou non-crossovers, entraînant respectivement l'échange des régions
flanquantes ou non \cite{mancera_high-resolution_2008}.

\begin{transition}
La réparation des cassures est la fonction principale et \emph{première} de la
machinerie de recombinaison homologue. Cependant, les mécanismes en jeu sont le
lieu d'un brassage génétique, aussi bien lors de la méiose eucaryote que lors
des transferts de gène procaryotes \cite{redfield_bacteria_2001}.
\end{transition}

\subsubsection{La recombinaison méiotique : étape clé de la méiose}
\label{sec:orgheadline4}

Chez les eucaryotes, la méiose implique la formation de DSB sous contrôle
génétique rigoureux, qui sont réparés par recombinaison homologue
\cite{chapman_playing_2012}. Les enzymes Spo11 introduisent aléatoirement des
\ac{dsb}. Cependant, la distribution des sites de coupure est variable : il
existe des \emph{hotspots} de cassure, et donc de recombinaison. Par opposition, les
\emph{coldspots} sont des régions moins soumises que d'autres aux cassures.

La réparation de ces \ac{dsb} par recombinaison homologue est requise pour
l'appariement et la ségrégation des chromosomes homologues au cours de la
méiose. Selon le mode de clivage des jonctions de Holliday par les résolvases,
des crossovers se forment entre les chromosomes parentaux. 

Ces crossovers entraînent le brassage des allèles, un processus bénéfique sur le
plan évolutif\cite{webster_direct_2012}. En effet, il casse les liaisons entre
allèles : la sélection élimine alors plus efficacement les variants délétères et
promeut les variants bénéfiques \cite{otto_resolving_2002}. C'est l'une des
hypothèses permettant d'expliquer l'évolution de la reproduction
sexuée\cite{otto_why_2006}.

\subsubsection{La recombinaison comme moteur des transferts horizontaux de gènes}
\label{sec:orgheadline5}

Étant donné la taille des populations bactériennes et les temps évolutifs en
jeu, la recombinaison a un impact majeur sur l'évolution procaryote
\cite{didelot_impact_2010}. C'est le moteur des transferts de gène. Ceux-ci sont
médiés soit par des vecteurs, les plasmides ou les phages, soit par un état de
compétence naturelle, \emph{via} l'acquisition passive ou active d'ADN exogène. 

La principale fonction de la recombinaison \emph{homologue} semble être la réparation
des lésions de l'ADN \cite{fall_horizontal_2007, michod_adaptive_2008}. 

Il est difficile d'estimer les fréquences de recombinaison procaryotes. Elles
sont très variables, au sein d'une espèce, d'un écotype ou entre espèces
\cite{didelot_impact_2010}.

\todo{Revoir cette partie}

\begin{transition}
Après la résolution des intermédiaires de recombinaison, des mésappariements
peuvent exister entre les différents brins. Leur correction entraîne une
conversion génique.
\end{transition}

\addfig{%
  \centering
  \includegraphics[width=\linewidth]{img/mutslh.pdf}
  \caption{\textbf{Correction des mésappariements par MutSLH}}
  \label{mutslh}
}

\subsection{Conversion génique}
\label{sec:orgheadline2}
La conversion génique est l'échange non réciproque d'information génétique.
C'est une transmission non-mendélienne : l'un des allèles a une plus forte
probabilité d'être transmis que l'autre\cite{chen_gene_2007}. 

Considérons le cas de la transmission de l'allèle \(A\) et de son homologue \(a\),
au cours de la méiose. Après la méiose, le génotype attendu est \(AAaa\).
Cependant, un évènement de conversion de gène peut conduire à des génotypes de
type \(Aaaa\) ou \(AAAa\). 

Au cours de la réparation des DSB, la conversion peut subvenir de deux façons.
i) L'allèle \(A\) est proche du site d'initiation de la cassure. Il fait partie de
la résection, l'allèle \(a\) est copié vers le brin réparé. \(Aaaa\) est le génotype
obtenu. ii) L'intermédiaire de recombinaison présente un polymorphisme \(Aa\) sur
l'un des hétéroduplex. La machinerie de réparation des mésappariements ---
\ac{mmr} --- les prend en charge. \(a\) est alors converti en \(A\), ou
réciproquement.

Chez \emph{E.coli}, la détection des mésappariements est effectuée par les dimères
des enzymes MutS. Les mésappariements sont reconnus par la distorsion qu'ils
causent à la structure de l'ADN. Les enzymes MutL et MutH sont alors recrutées.
Une cassure est introduite dans l'un des brins, suivie par une résection souvent
supérieure à \$1\$kb à proximité de la cassure. Une ADN polymérase utilise ensuite
le brin intact pour synthétiser la région complémentaire. Les eucaryotes
possèdent des protéines aux fonctions homologues, appelées respectivement MSH et
MLH pour \emph{MutS Homologs} et \emph{MutL Homologs}. Ce sont des composants de la voie
\ac{ner}, \emph{Nucleotide Excision Repair}.

Au cours de la recombinaison, le système de \ac{mmr} est la voie préférentielle
de correction des mésappariements dans l'hétéroduplex. Néanmoins, la voie
\ac{ber}, \emph{Base Excision Repair}, est une alternative à ce système.

Elle entraîne l'excision de l'une des bases du mésappariement, puis son
remplacement par la base complémentaire à l'autre. Les ADN glycosylases excisent
les bases avec une spécificité de substrat : chaque base \(A\), \(T\), \(C\) ou \(G\) a
une ADN glycosylase correspondante et spécifique.

Dans tous les cas, le génotype de la région --- ou de la base --- digérée est
\emph{converti} par celui du brin intact. Le transfert a lieu entre séquences
homologues, qu'elles soient sur des chromatides sœurs, sur le même chromosome ou
sur des chromosomes différents \cite{chen_gene_2007}.

\begin{transition}

En théorie, la conversion $a \mapsto A$ a lieu avec la même fréquence que celle
de la conversion $A \mapsto a$. Cependant, dès lors qu'un allèle est plus
souvent converti que l'autre, à l'échelle de la population, la conversion
génique est {\em biaisée}. Chez les eucaryotes, de nombreuses observations montrent
que les mésappariements $GA$, $GT$, $CA$ ou $CT$ sont plus fréquemment corrigés
en $GC$ qu'en $AT$ \cite{duret_biased_2009}. 

\end{transition}

\subsection{La conversion génique biaisée vers GC}
\label{sec:orgheadline7}

\todo{premiers articles gbgc marais et duret 2003 et 2001 etc}

Mancera \emph{et al} \cite{mancera_high-resolution_2008} ont génotypé l'ensemble des
quatre haplotypes --- les tétrades --- résultants des produits de méiose de 46
levures, à haute résolution. Ils montrent qu'1\% du génome de chaque produit de
méiose est soumis à de la conversion génique. Ces régions montrent une
transmission biaisée en faveur des allèles G ou C. Ils sont transmis avec une
probabilité 1.3\% plus élevée qu'attendu sous l'hypothèse d'une transmission
mendélienne \cite{mancera_high-resolution_2008}. 

Chez la levure, le \ac{gbgc} est associé spécifiquement aux produits de
recombinaisons entraînant des crossovers \cite{lesecque_gc-biased_2013}. Il est
également associé aux évènements de conversion \emph{simple} --- par opposition aux
évènements complexes. Lors d'un évènement de conversion simple, le même brin est
le donneur de la conversion sur l'ensemble de la région convertie. Lors d'un
évènement complexe, les deux brins de l'hétéroduplex peuvent être donneur. 

\begin{transition}
  Les causes moléculaires de l'existence d'un tel mécanisme suscitent beaucoup
  d'interrogations. Différentes hypothèses ont été avancées : elles font l'objet
  de la partie suivante. 
\end{transition}

\section{Quelles hypothèses pour l'expliquer ?}
\label{sec:orgheadline11}
\subsection{Des propriétés inhérentes à la machinerie de réparation ?}
\label{sec:orgheadline9}
La machinerie de réparation pourrait dans sa structure présenter un biais en
faveur de la transmission des allèles G ou C au cours de la conversion génique. 
\subsection{Un processus sélectionné pour compenser la mutation ?}
\label{sec:orgheadline10}
La mutation est universellement biaisée vers AT
\cite{lynch_rate_2010,hershberg_evidence_2010}. Le gBGC pourrait avoir été
sélectionné pour contrecarrer les effets de ce biais
mutationnel\cite{marais_biased_2003}. Des preuves théoriques et expérimentales
existent pour soutenir cette hypothèse.

La voie \ac{ber} implique des DNA glycosylases spécifiques. Chez les mammifères,
il existe des Thymines DNA glycosylases, qui excisent les bases thymines. 
\section{Quelles en sont les conséquences ?}
\label{sec:orgheadline17}
\subsection{Structure le contenu en GC}
\label{sec:orgheadline14}
\subsubsection{Le contenu en GC, un trait sous sélection ?}
\label{sec:orgheadline12}
\subsubsection{La théorie des isochores}
\label{sec:orgheadline13}
\subsection{Interfère avec la sélection}
\label{sec:orgheadline15}
\subsection{Brouille les tests de sélection}
\label{sec:orgheadline16}
\section*{Conclusion}
\label{sec:orgheadline18}
\newpage
\setstretch{1.1}
\bibliography{references}
\newpage
\pagenumbering{roman}
\setcounter{page}{1}

\end{document}