\newpage
\setstretch{1.0}
\bibliography{references}

\newpage
\pagenumbering{roman}
\setcounter{page}{1}
\setcounter{lof}{1}

\setcounter{section}{0}
\section{Annexes}
\setcounter{subsection}{1}

% \subsubsection*{Interférences entre biais et sélection}

% \setstretch{1.5}
% Le destin d'un allèle soumis à de la conversion génique biaisée est déterminée
% par deux paramètres principaux : $S$ l'intensité de la pression de sélection à
% laquelle elle est soumise, et $B$ l'intensité de l'action du biais de
% conversion. Ces deux paramètres dépendent de $Ne$ la taille efficace de la
% population. 
% \begin{align}
%   S &= N_e \times s \\
%   B &= N_e \times b \\
% \end{align}

% $N_e$ étant un facteur commun, le rapport entre les deux forces déterminantes
% n'est fonction que de $s$ et de $b$, l'intensité du biais. 

% \newpage
\subsection{Correction des mésappariements : le modèle {\em E.coli}}
\begin{figure*}[htbp]
  \centering
  \includegraphics[width=0.6\linewidth]{img/mutslh.pdf}
  \caption*{{\bf La réparation des mésappariements par le complexe MutSLH.}
    \rmfamily Le dimère de MutS reconnaît l’anomalie structurelle de l’ADN causé
    par un mésappariement. Il recrute alors l’enzyme MutL, qui à son tour
    recrute l’endonucléase MutH. Celle-ci introduit une cassure sur l’un des
    brins. Elle est suivie d’une résection par une exonucléase, puis par la
    synthèse {\em via} une ADN polymérase, sur la base du brin intact. Chez les
    eucaryotes, l'homologue de MutH est absent. C'est la présence d'une cassure
    qui détermine le choix du brin matrice.\\
    {\em Adapté de Molecular Biology of The Gene, Watson, 2012.}}
  \label{mutslh}
\end{figure*}

\newpage

\subsection{Les arcs et les pendentifs}
\label{sub:arcs}
\begin{figure}[htbp]
  \centering
  \includegraphics[width=0.8\linewidth]{img/arcs.pdf}
  
  \caption*{ {\bf Les dômes de la basilique Saint Marc,} \rmfamily%
    à Venise, reposent chacun sur quatre arcs. Une telle architecture implique
    l'apparition de pendentifs, qui ont été exploités par les décorateurs. En
    1979, Stephen Jay Gould et Richard Lewontin ont utilisé cette image pour
    prévenir des dérives adaptationnistes\cite{gould_spandrels_1979}. Les
    pendentifs ne sont que les effets secondaires de l'adoption de
    l'architecture à quatre arcs. Ce serait une erreur de penser que les
    pendentifs étaient activement recherchés par les architectes. De même, dans
    le Vivant, de nombreux caractères ne sont que les résultantes de pressions
    de sélections qui agissent à d'autres niveaux, sur d'autres caractères. Le
    brassage génétique induit par la recombinaison homologue est-il le pendentif
    de la réparation de l'ADN \cite{fall_horizontal_2007}? La conversion génique
    biaisée est-elle le pendentif de la recombinaison homologue ? }
  \label{}
\end{figure}

