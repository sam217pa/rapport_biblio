% Created 2015-11-06 Fri 12:40
\documentclass[11pt]{scrartcl}
\usepackage[scaled]{helvet}
\usepackage{hyperref}
\renewcommand\familydefault{\sfdefault}
\usepackage[a4paper]{geometry}
\geometry{
  a4paper,
  left=25mm,
  right=25mm,
  top=25mm,
  bottom=25mm
}

\linespread{1.5}

\usepackage[utf8]{inputenc}
\usepackage[T1]{fontenc}
\usepackage[french, english]{babel}
\usepackage{fixltx2e}
\usepackage{graphicx}
\usepackage{longtable}
\usepackage{float}
\usepackage{hyperref}
\usepackage{wrapfig}
\usepackage{rotating}
\usepackage[normalem]{ulem}
\usepackage{amsmath}
\usepackage{textcomp}
\usepackage{marvosym}
\usepackage{wasysym}
\usepackage{amssymb}
\usepackage[scaled=0.9]{zi4}
\usepackage[usenames]{xcolor}
\usepackage[protrusion=true, expansion=alltext, tracking=true, kerning=true]{microtype}
\usepackage{siunitx}
\usepackage{csquotes}
\usepackage[authoryear, round]{natbib}
\bibliographystyle{asm}
\setcitestyle{authoryear, open={(}, close={)}}
\author{Samuel Barreto}
\date{\today}
\title{Rapport Bibliographique}
\hypersetup{
  pdfkeywords={},
  pdfsubject={},
  pdfcreator={Emacs 24.5.1 (Org mode 8.2.10)}}
\begin{document}

\maketitle

\section{{\bfseries\sffamily TODO} Introduction}
\label{sec-1}
\section{{\bfseries\sffamily TODO} Qu'est-ce que la conversion génique ? [0\%]}
\label{sec-2}

La conversion génique est un mécanisme qui intervient au cours de la réparation
de l'ADN. Lors d'une cassure double brin, la lésion est réparée par
recombinaison homologue. Les mésappariemments induits peuvent être corrigés,
donnant lieu à une transmission d'allèle non-mendélienne. 

\subsection{{\bfseries\sffamily ->>-} La recombinaison homologue}
\label{sec-2-1}

\subsubsection{Quoi et où ?}
\label{sec-2-1-1}
La recombinaison homologue est un mécanisme ubiquitaire. Son fonctionnement est
très conservé à la fois chez les eucaryotes et chez les procaryotes. C'est le
mécanisme essentiel de réparation de l'ADN. Il corrige les cassures double-brins
et les erreurs de fourches de réplications \cite{lusetti_bacterial_2002}. 


\subsubsection{Brassage génétique}
\label{sec-2-1-2}
C'est un processus de brassage génétique clé \cite{webster_direct_2012}. Chez les
eucaryotes, la méiose implique une étape de recombinaison, sous contrôle
génétique strict. Chez les procaryotes, c'est le moteur des transferts
horizontaux de gènes \cite{didelot_impact_2010}.

\subsubsection{Moteur de la sélection}
\label{sec-2-1-3}
La recombinaison rend la Sélection plus efficace. En l'absence de recombinaison,
les locus sous sélection interfèrent entre eux. Ce sont les interférences de
Hill et Robertson \cite{hill_effect_1966}. 

\subsection{{\bfseries\sffamily TODO} La conversion génique}
\label{sec-2-2}
\subsection{{\bfseries\sffamily TODO} La conversion génique biaisée vers GC}
\label{sec-2-3}
\section{{\bfseries\sffamily TODO} Comment ça marche ?}
\label{sec-3}
\section{{\bfseries\sffamily TODO} Quelles en sont les conséquences ?}
\label{sec-4}
\section{{\bfseries\sffamily TODO} Quelles hypothèses pour l'expliquer ?}
\label{sec-5}
\section{{\bfseries\sffamily TODO} Conclusion}
\label{sec-6}

\linespread{0.8}
\bibliography{references}
% Emacs 24.5.1 (Org mode 8.2.10)
\end{document}
